\documentclass[12pt,a4paper]{article}
\usepackage[utf8]{inputenc}
\usepackage[T1]{fontenc}
\usepackage{amsmath}
\usepackage{amssymb}
\usepackage{graphicx}
\title{Relációs algebra}

\begin{document}
	\maketitle	
	
	$ \pi $ projekció
	$ \sigma $ szelekció
	$ \rho $ átnevezés
	
	\begin{tabular}{|c|c|}
		\hline
		Név & Gyümölcs \\
		\hline
		Micimackó & Alma \\
		\hline
		Micimackó & Körte \\
		\hline
		Tigris & Alma \\
		\hline
		Bagoly & Eper \\
		\hline
	\end{tabular}

	Adjuk meg azokat a gyumolcsoket melyeket micimackó sezret!
	
	\[
		\pi_{gyumolcs}{\left(  \sigma_{nev = micimacko}{\left( SZERET  \right) }\right) }
	\]
	
	Minden reláció egy újabb relációt ad vissza, ezért tudunk így láncolni.
	
	Értlemezve van a: $  \cup, \; \cap $ és komplementer műveletek.
	
	Adjuk meg azokat a gyümölcsöket melyeket miciamckó nem szeret
	
	\[
		\pi_{gyumolcs}{SZERET} \setminus \pi_{gyumolcs}{\left(  \sigma_{nev = micimacko}{\left( SZERET  \right) }\right) }
	\]
\end{document}