\documentclass[a4paper,12pt]{article}

\usepackage[margin=0.75in]{geometry}
\usepackage[utf8]{inputenc}
\usepackage{exsheets}

\DeclareInstance{exsheets-heading}{block-no-nr}{default}{
	attach = {
		main[l,vc]title[l,vc](0pt,0pt) ;
		main[r,vc]points[l,vc](\marginparsep,0pt)
	}
}
\RenewQuSolPair
{question}[headings=runin]
{solution}[headings=block-no-nr]

\SetupExSheets{
	counter-format=qu.,
	solution/print=true ,
	question/name=Feladat,
	solution/name=Megoldás.
}

\usepackage[hungarian]{babel}
\usepackage{amsmath}
\usepackage{mathtools}
\usepackage{amsthm}
\usepackage{amsfonts}
\usepackage{amssymb}
\usepackage{graphicx}
\usepackage{wrapfig}
\graphicspath{{./images/}}
\usepackage{float}
\usepackage{multicol}
\usepackage{icomma}

\usepackage{titling}
\setlength{\droptitle}{-2cm}

\title{\huge{Numerikus módszerek} \\[-4pt] \large 1. gyakorlat \vspace{-15pt}}
\author{Boda Bálint}
\date{\vspace{-12pt}2023. tavaszi félév}

\DeclareMathOperator{\lf}{lf}

\theoremstyle{definition}
\newtheorem{definition}{Definíció}
\newtheorem*{definition*}{Definíció}
\newtheorem*{remark}{Megjegyzés}
\newtheorem{theorem}{Tétel}
\newtheorem*{theorem*}{Tétel}
\newtheorem*{example}{Példa}
\newtheorem{notation}{Jelölés}
\newtheorem*{notation*}{Jelölés}

\SetupExSheets{solution/print=true}
\SetupExSheets{question/name=}
\SetupExSheets{headings=runin}

\begin{document}
	\maketitle
	\vspace{-10pt}
	\section{A lebegőpontos számábrázolás egy modellje}
		Alapötlet: tároljuk el a számokat ún. normalizált alakban. Például: $ 324 \rightsquigarrow 0,324 \cdot 10^3 $ Kettes számrendszerben: $ +0,101000100 \cdot 2^9 $.
	\begin{definition*}[Normalizált lebegőpontos szám]
		Legyen $ t \in \mathbb{N} $ (bitek száma) és $ m = \sum\limits_{i=1}^{t}{m_i \cdot 2^{i-1}} $, ahol $ m_1 = 1, \; m_i \in \left\lbrace 0, 1 \right\rbrace $. Ekkor normalizált lebegőpontos számnak nevezzük az
		\[
		a = \pm m \cdot 2^k \; (k \in \mathbb{Z})
		\]
		alakú számokat. Az $m$ számot az a szám mantisszájának a $k$ számot az a szám karakterisztikájának nevezzük.
	\end{definition*}

	\begin{notation*}
		$ a = \pm \left[ m_1 \dots m_t | k \right] $
	\end{notation*}
\end{document}