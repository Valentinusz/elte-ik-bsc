\documentclass[a4paper,12pt]{article}

\usepackage[margin=1in]{geometry}
\usepackage[utf8]{inputenc}
\usepackage{exsheets}
\usepackage{centernot}
\usepackage{listings}

\DeclareInstance{exsheets-heading}{block-no-nr}{default}{
    attach = {
        main[l,vc]title[l,vc](0pt,0pt) ;
        main[r,vc]points[l,vc](\marginparsep,0pt)
    }
}
\RenewQuSolPair
{question}[headings=runin]
{solution}[headings=block-no-nr]

\SetupExSheets{
    counter-format=qu.,
    solution/print=true ,
    question/name=Feladat,
    solution/name=Megoldás.
}

\usepackage{tasks}
\usepackage[hungarian]{babel}
\usepackage{amsmath}
\usepackage{mathtools}
\usepackage{amsthm}
\usepackage[shortlabels]{enumitem}
\usepackage{amsfonts}
\usepackage{amssymb}
\usepackage{graphicx}
\usepackage{wrapfig}
\graphicspath{{./images/}}
\usepackage{float}
\usepackage{multicol}
\usepackage{tikz}
\usepackage{booktabs}
\usepackage{lstmisc}
\usepackage{cancel}
\usepackage{mathtools}
\usetikzlibrary{positioning,shapes,fit,arrows}
\tikzset{every picture/.style={line width=0.75pt}} %set default line width to 0.75pt

\title{\huge{Analízis II} \\ \large 8. Házi feladat}
\author{Boda Bálint}
\date{2022. őszi félév}

\theoremstyle{definition}
\newtheorem{definition}{Definíció}
\newtheorem*{definition*}{Definíció}
\newtheorem*{remark}{Megjegyzés}
\newtheorem{theorem}{Tétel}
\newtheorem*{theorem*}{Tétel}
\newtheorem*{example}{Példa}

\DeclareMathOperator{\lf}{lf}
\DeclareMathOperator{\ps}{p(S)}
\DeclareMathOperator{\prim}{pr\acute \jmath m}
\DeclareMathOperator{\tg}{tg}
\DeclareMathOperator{\arctg}{arctg}

\SetupExSheets{solution/print=true}
\SetupExSheets{question/name=}
\SetupExSheets{headings=runin}

\begin{document}
	\maketitle
	\begin{question}
		Számítsuk ki a következő határozatlan integrálokat!
		\begin{tasks}(2)
			\task \[ \int{\frac{x^3+x^2-x+3}{x^2+x-2} \mathop{dx}} \quad (x > 1) \]
			\task \[ \int{\frac{x^4-x^2+1}{x^2(x+1)} \mathop{dx}} \quad (0 < x < 1)\]
			\task \[ \int{\frac{x+1}{x^2+3x+4} \mathop{dx}} \quad (x \in \mathbb{R}) \]
			\task \[ \int{\frac{2x^2+x+1}{x^2(x^2+1)} \mathop{dx}} \quad (x > 0) \]
		\end{tasks}
	\end{question}
	\newpage
	\begin{solution}
		\begin{tasks}
			\task{ Mivel a számláló fokszáma nagyobb mint a nevezőé először fel kell bontanunk a törtet egy törtre és egy polinomra:
				\[
					\frac{x^3+x^2-x+3}{x^2+x-2} = x + \frac{x+3}{x^2+x-2} = x + \frac{x+3}{(x-1)(x+2)}
				\]
				Alakítsuk át a törtet a parciális törtekre hozás módszerével:
				\[
					\frac{x+3}{(x-1)(x+2)} = x + \frac{A(x+2)+B(x-1)}{(x-1)(x+2)} = x + \frac{A}{x-1} + \frac{B}{x+2}
				\]
				\\[-20pt]
				\[
					A(x+2) + B(x-1) = x + 3
				\]
				\\[-35pt]
				\begin{alignat*}{3}
					x = 1,&& \quad 3A = 4 &\implies A &= \frac{4}{3} \\
					x = -2,&& \quad -3B = 1 &\implies B &= -\frac{1}{3}
				\end{alignat*}
				Így a tört:
				\[
				\frac{x^3+x^2-x+3}{x^2+x-2} = x + \frac{4}{3(x-1)} - \frac{1}{3(x+2)}
				\]
				\begin{align*}
					\int{\frac{x^3+x^2-x+3}{x^2+x-2} \mathop{dx}} &= \int{x \mathop{dx}} + \int{\frac{4}{3(x-1)} \mathop{dx}} - \int{\frac{1}{3(x+2)} \mathop{dx}} \\
					&= \int{x \mathop{dx}} + \frac{4}{3} \int{\frac{1}{x-1} \mathop{dx}} - \frac{1}{3} \int{\frac{1}{x+2} \mathop{dx}} \\
					&= \frac{1}{2}x^2 + \frac{4}{3} \ln{(x-1)} - \frac{1}{3} \ln{(x+2)} + c 
				\end{align*}
			}
			\newpage
			\task { Mivel a számláló fokszáma nagyobb mint a nevezőé először fel kell bontanunk a törtet egy törtre és egy polinomra:
				\begin{align*}
					\frac{x^4-x^2+1}{x^2(x+1)} &= \frac{x^4+x^3-x^3-x^2+1}{x^3+x^2} = \frac{x(x^3+x^2)-x^3-x^2+1}{x^3+x^2} \\
					&= x +\frac{-x^3-x^2+1}{x^3+x^2} = x - \frac{x^3+x^2}{x^3+x^2} + \frac{1}{x^3+x^2} \\
					&= x - 1 + \frac{1}{x^3+x^2}
				\end{align*}
				Bontsuk parciális törtekre a $ \frac{1}{x^3+x^2} $ kifejezést:
				\begin{align*}
					\frac{1}{x^2(x+1)} = \frac{A}{x} + \frac{B}{x^2} + \frac{C}{x+1}
				\end{align*}
				A következő egyenletet kapjuk:
				\begin{align*}
					A \cdot x \cdot (x+1) + B \cdot (x+1) + C \cdot x^2 &= 1 \\
					Ax^2 + Ax + Bx + B + Cx^2 &= 1
				\end{align*}

				Ebből:
				\begin{align*}
					A + C &= 0 \iff -1 + C = 0 \iff C = 1 \\
					A + B &= 0 \iff A + 1 = 0 \iff A = -1 \\ 
					B &= 1
				\end{align*}
				Így a tört:
				\[
					\frac{1}{x^2(x+1)} = -\frac{1}{x} + \frac{1}{x^2} + \frac{1}{x+1}
				\]
				\begin{align*}
					\int{\frac{x^4-x^2+1}{x^2(x+1)} \mathop{dx}} &= \int{x \mathop{dx}} - \int{1 \mathop{dx}} - \int{\frac{1}{x} \mathop{dx}} + \int{\frac{1}{x^2} \mathop{dx}} + \int{\frac{1}{x+1} \mathop{dx}} \\
					&= \frac{1}{2}x^2 - x - \ln{(x)} - \frac{1}{x} + \ln{(x+1)} + c
				\end{align*}
			}
			\task{
				Bontsuk fel a törtet a következő módon:
				\begin{align*}
				\frac{x+1}{x^2+3x+4} = \frac{\gamma((x^2+3x+4)')}{x^2+3x+4} + \frac{\delta}{x^2+3x+4} = \frac{\gamma(2x+3)}{x^2+3x+4} + \frac{\delta}{x^2+3x+4} 
				\end{align*}
				Ebből a következő egyenlet adódik:
				\[
				1x+1 = \gamma(2x+3) + \delta \\
				\]
				Ami akkor teljesül ha:
				\begin{align*}
					x &= \gamma \cdot 2x \iff \gamma = \frac{1}{2} \\
					1 &= \gamma \cdot 3 + \delta = \frac{3}{2} + \delta \iff \delta = -\frac{1}{2}
				\end{align*}
				Ez alapján a tört:
				\begin{align*}
					\frac{x+1}{x^2+3x+4} = \frac{\frac{2x+3}{2}-\frac{1}{2}}{x^2+3x+4} = \frac{2x+3}{2(x^2+3x+4)} - \frac{1}{2(x^2+3x+4)}
				\end{align*}
				Így:
				\begin{align*}
					\int{\frac{x+1}{x^2+3x+4} \mathop{dx}} = \frac{1}{2} \int{\frac{2x+3}{x^2+3x+4} \mathop{dx}} - \frac{1}{2} \int{\frac{1}{x^2+3x+4} \mathop{dx}}
				\end{align*}
				ahol,
				\begin{align*}
					\int{\frac{1}{x^2+3x+4} \mathop{dx}} &= \int{\frac{1}{\left(x+\frac{3}{2}\right)^2 + \frac{7}{4}} \mathop{dx}} = \frac{7}{4} \int{\frac{1}{\left( \sqrt{\frac{4}{7}} \left(x+\frac{3}{2}\right)\right)^2 + 1 } \mathop{dx}} \\
					&= \frac{4}{7} \cdot \frac{\arctg{\left( \sqrt{\frac{4}{7}} \left( x+\frac{3}{2} \right) \right)}}{\sqrt{\frac{4}{7}}}
				\end{align*}
			
				\[
					\int{\frac{x+1}{x^2+3x+4} \mathop{dx}} = \frac{1}{2} \cdot \ln{|x^2+3x+4|} - \frac{2}{7} \cdot \frac{\arctg{\left( \sqrt{\frac{4}{7}} \left( x+\frac{3}{2} \right) \right)}}{\sqrt{\frac{4}{7}}} + c
				\]
			}
			\task{
				Bontsuk fel a kifejezést a parciális törtekre bontás módszerével:
				\begin{align*}
					\frac{2x^2+x+1}{x^2(x^2+1)} = \frac{A}{x} + \frac{B}{x^2} + \frac{Cx+D}{x^2+1}
				\end{align*}
				Ebből a következő egyenlet adódik:
				\[
				A(x)(x^2+1) + B(x^2+1) + (Cx+D)(x^2) = Ax^3 + Ax + Bx^2 + B + Cx^3 + Dx^2
				\]
				Ez pontosan akkor teljesül, ha:
				\[
				Ax^3 + Ax + Bx^2 + B + Cx^3 + Dx^2 = 2x^2 + x + 1
				\]
				Melyből a következő egyenletrendszer adódik:
				\begin{align*}
					A &= 1 \\
					B &= 1 \\
					A + C &= 0 \iff 1 + C = 0 \iff C = -1 \\
					B + D &= 2 \iff 1 + D = 2 \iff D = 1 
				\end{align*}
				Így a felbontott tört a következő:
			\begin{align*}
				\frac{2x^2+x+1}{x^2(x^2+1)} = \frac{1}{x} + \frac{1}{x^2} - \frac{x-1}{x^2+1}
			\end{align*}
		
			\begin{align*}
				\int{\frac{2x^2+x+1}{x^2(x^2+1)} \mathop{dx}} &= \int{\frac{1}{x} \mathop{dx}} + \int{\frac{1}{x^2} \mathop{dx}} - \int{\frac{x}{x^2+1} \mathop{dx}} - \int{\frac{1}{x^2+1} \mathop{dx}} \\
				&= \int{\frac{1}{x} \mathop{dx}} + \int{\frac{1}{x^2} \mathop{dx}} - \frac{1}{2} \cdot \int{\frac{2x}{x^2+1} \mathop{dx}} - \int{\frac{1}{x^2+1} \mathop{dx}} \\
				&= \ln{|x|} - \frac{1}{x} - \frac{1}{2} \ln{x^2+1} - \arctg{x} + c
			\end{align*}
			}		
		\end{tasks}
	\end{solution}
\end{document}