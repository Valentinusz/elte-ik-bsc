\documentclass[a4paper,12pt]{article}

\usepackage[margin=0.75in]{geometry}
\usepackage[utf8]{inputenc}
\usepackage{exsheets}

\DeclareInstance{exsheets-heading}{block-no-nr}{default}{
    attach = {
        main[l,vc]title[l,vc](0pt,0pt) ;
        main[r,vc]points[l,vc](\marginparsep,0pt)
    }
}
\RenewQuSolPair
{question}[headings=runin]
{solution}[headings=block-no-nr]

\SetupExSheets{
    counter-format=qu.,
    solution/print=true ,
    question/name=Feladat,
    solution/name=Megoldás.
}

\usepackage[hungarian]{babel}
\usepackage{amsmath}
\usepackage{mathtools}
\usepackage{amsthm}
\usepackage{amsfonts}
\usepackage{amssymb}
\usepackage{graphicx}
\usepackage{wrapfig}
\graphicspath{{./images/}}
\usepackage{float}
\usepackage{multicol}
\usepackage{stuki}

\usepackage{titling}
\setlength{\droptitle}{-2cm}

\title{\huge{Analízis II.} \\[-4pt] \large 11. házi feladat \vspace{-15pt}}
\author{Boda Bálint}
\date{\vspace{-12pt}2022. őszi félév}

\SetupExSheets{solution/print=true}
\SetupExSheets{question/name=}
\SetupExSheets{headings=runin}

\begin{document}
    \maketitle
    \vspace{-10pt}
\begin{question}
	A definíció alapján bizonyítsa be, hogy az 
	\[
		f(x,y) \coloneq x^3+xy \quad \bigl( (x,y) \in \mathbb{R}^2 \bigr)
	\]
	függvény totálisan deriválható az $a \coloneq (2,3) $ és adja meg  az $f'(a)$ deriváltmátrixot! Az $f'(a)$-ra kapott eredményt ellenőrizze a Jacobi-mátrix kiszámításával!
\end{question}
\begin{solution}
	Legyen $ a = (a_1,a_2) = (2,3) $ és $ h = (h_1,h_2) \in \mathbb{R}^2 $. Azt kell belátnunk, hogy van olyan ${ A = \begin{pmatrix} A_1 & A_2 \end{pmatrix} \in \mathbb{R}^{1 \times 2}}$ sormátrix, amire:
	\begin{align*}
		\lim_{h \rightarrow 0}{\frac{\left|  f(a+h) - f(a) - A \cdot h\right|}{\lVert h \rVert}} =
		\lim_{(h_1,h_2) \rightarrow (0,0)}{\frac{ \left| f(a+h) - f(a) - \begin{pmatrix} A_1 & A_2 \end{pmatrix} \cdot \begin{pmatrix} h_1 \\ h_2 \end{pmatrix} \right| }{\sqrt{h_1^2 + h_2^2}}} = 0
	\end{align*}
	Határozzuk meg az $A$ mátrixot:
	\begin{align*}
		f(a+h) - f(a) &= f(a_1+h_1,a_2+h_2) - f(a_1,a_2) = f(2+h_1,3+h_2) - f(2,3) = \\
		&= (2+h_1)^3 + (2+h_1)(3+h_2) - (2^3 + 2 \cdot 3 ) = \\
		&= 8 + 12h_1 + 6h_1^2 + h_1^3 + 6 + 2h_2 + 3h_1 + h_1h_2 - 14 = \\
		&= 15h_1 + 2h_2 + 6h_1^2 + h_1^3 + h_1h_2 = \\
		&= \begin{pmatrix} 15 & 2 \end{pmatrix} \cdot \begin{pmatrix} h_1 \\ h_2 \end{pmatrix} + 6h_1^2 + h_1^3 + h_1h_2 
	\end{align*}
	így:
	\begin{align*}
		\frac{ \left| f(a+h) - f(a) - \begin{pmatrix} A_1 & A_2 \end{pmatrix} \cdot \begin{pmatrix} h_1 \\ h_2 \end{pmatrix} \right| }{\sqrt{h_1^2 + h_2^2}} = \frac{ \left| h_1^3 + 6h_1^2 + h_1h_2 \right| }{\sqrt{h_1^2 + h_2^2}}
	\end{align*}
	Azt kell bizonyítani, hogy a jobb oldali függvény határértéke a $ (0,0) $ pontban 0.
	\[
	\forall \varepsilon > 0, \exists \delta > 0, \forall (h_1,h_2) \in D_f \setminus \{(0,0)\}: 0 < \lVert (h_1,h_2) \rVert < \delta: \left | \frac{ \left| h_1^3 + 6h_1^2 + h_1h_2 \right| }{\sqrt{h_1^2 + h_2^2}} \right| < \varepsilon
	\]
	\begin{align*}
		\left| \frac{ \left| h_1^3 + 6h_1^2 + h_1h_2 \right| }{\sqrt{h_1^2 + h_2^2}} \right| = \frac{ \left| h_1^3 + 6h_1^2 + h_1h_2 \right| }{\sqrt{h_1^2 + h_2^2}} = \frac{ \left| h_1^3 \right| + 6h_1^2 + \left| h_1h_2 \right| }{\sqrt{h_1^2 + h_2^2}} < \varepsilon
	\end{align*}
	Ha feltesszük, hogy $\sqrt{h_1^2 + h_2^2} < 1$, akkor $|h_1| < 1$ és így $ \left| h_1^3 \right| \le h_1^2 $. Továbbá tudjuk, hogy $ |h_1h_2| \le \frac{1}{2}(h_1^2 + h_2^2) $ Így:
	\begin{align*}
		\frac{ \left| h_1^3 \right| + \left| h_1h_2 \right| }{\sqrt{h_1^2 + h_2^2}} \le \frac{7h_1^2 + \frac{1}{2}(h_1^2 + h_2^2) }{\sqrt{h_1^2 + h_2^2}} \le \frac{8h_1^2+8h_2^2 }{\sqrt{h_1^2 + h_2^2}} = 8\sqrt{h_1^2 + h_2^2} < \varepsilon
	\end{align*}
	Így, ha $ \delta \coloneq \min{\left( 1,\frac{\varepsilon}{8} \right)} $ akkor az előbbi függvény határértéke $0$ a $(0,0)$ pontban, így $f \in D{(2,3)} $ és $f'(a) = \begin{pmatrix} 15 & 2 \end{pmatrix} $.
	\newpage \noindent
	Ellenőrizzük, a kapott eredményt a Jacobi-mátrix meghatározásával:
	\begin{align*}
		\partial_1 f(x,y) &= 3x^2 + y &&\partial_1 f(2,3) = 12 + 3 = 15 \\
		\partial_2 f(x,y) &= x &&\partial_2 f(2,3) = 2
	\end{align*}
	Így a Jacobi-mátrix: $ \begin{pmatrix} 15 & 2 \end{pmatrix} $, ami valóban megegyezik a kapott eredménnyel.
\end{solution}
\vspace{40px}
\begin{question}
	Írja fel a $ z = x^{2} \cdot e^{xy} $ egyenletű felület $P_0(1,0)$ pontjához tartozó érintősíkjának egyenletét, és adja meg egy normálvektorát!
\end{question}
\begin{solution}
	Világos, hogy léteznek a parciális deriváltak:
	\begin{align*}
		\partial_{1}{f(x,y)} &= 2x \cdot e^{xy} + x^2 \cdot e^{xy} \cdot y   &\partial_{1}{f(1,0)}=2 \\
		\partial_{2}{f(x,y)} &= x^2 \cdot e^{xy} \cdot x = x^3 \cdot e^{xy}  &\partial_{2}{f(1,0)}=1
	\end{align*}
	Mivel a parciális deriváltak léteznek $P_0 = (x_0,y_0)$ pont egy környezetében és folytonosak a pontban, ezért $f$ totálisan deriválható a pontban. A felület $(x_0,y_0,f(x_0,y_0))$ pontjához a következő érintősík húzható:
	\begin{align*}
		z - f(x_0,y_0) &= \partial_{1}{f(x_0,y_0)}(x-x_0) + \partial_{2}{f(x_0,y_0)}(y-y_0) \\
		z - 1 &= 2(x-1) + y \\
		1 &= 2x + y - z
	\end{align*}
	Ennek egy normálvektora: $(2,1,-1)$.
\end{solution}
\newpage
\begin{question}
	Határozza meg az
	\[
	f(x,y) \coloneq 2x^3 - 6x + y^3 - 12y + 5 \quad \bigl( (x,y) \in \mathbb{R}^2 \bigr)
	\]
	függvény lokális szélsőértékhelyeit.
\end{question}
\begin{solution}
	Az $f$ függvény egy kétváltozós polinom, ezért $f \in C^2{(\mathbb{R}^2)} $.  Az elsőrendű szükséges feltétel alapján:
	\begin{align*}
		\partial_{1}{f(x,y)} &= 6x^2 - 6  = 0 &x_{1,2}= \pm 1 \\
		\partial_{2}{f(x,y)} &= 3y^2 - 12 = 0 &y_{1,2}= \pm 2
	\end{align*}
	A stacionárius pontok a következők:
		\[
		P_1(1,2), \quad P_2(-1,2), \quad P_3(1,-2), \quad P_4(-1,-2) 
		\]
	Határozzuk meg a Hesse-féle mátrixot:
		\[
			\partial_{xx}{f(x,y)} = 12x, \quad \partial_{xy}{f(x,y)} = 0, \quad \partial_{yx}{f(x,y)} = 0, \quad \partial_{yy}{f(x,y)} = 6y
		\]
		\begin{equation*}
			f''(x,y) =
			\begin{pmatrix}
				\partial_{xx}{f(x,y)} & \partial_{xy}{f(x,y)} \\
				\partial_{yx}{f(x,y)} & \partial_{yy}{f(x,y)}
			\end{pmatrix} =
			\begin{pmatrix}
				12x & 0 \\
				0 & 6y
			\end{pmatrix}
		\end{equation*}
		\[
		D_1 = 12x, \quad D_2 = \det f''(x,y) = 12x \cdot 6y = 72xy
		\]
		Vizsgáljuk meg a stacionárius pontokat:
		\begin{enumerate}
			\item $P_1(1,2)$
			\begin{equation*}
				f''(1,2) =
				\begin{pmatrix}
					12 & 0 \\
					0 & 12
				\end{pmatrix}, D_2 = 72 \cdot 12 \cdot 12 > 0 \text{ és } D_1 = 12 > 0 
			\end{equation*}
			Így $f$-nek $P_1$-ben lokális minimumhelye van.
			\item $P_2(-1,2)$
			\begin{equation*}
				f''(-1,2) =
				\begin{pmatrix}
					-12 & 0 \\
					0 & 12
				\end{pmatrix}, D_2 = 72 \cdot -12 \cdot 12 < 0
			\end{equation*}
			Így $f$-nek $P_2$-ben nincs lokális szélsőértéke.
			\item $P_3(1,-2)$
			\begin{equation*}
				f''(1,-2) =
				\begin{pmatrix}
					12 & 0 \\
					0 & -12
				\end{pmatrix}, D_2 = 72 \cdot 12 \cdot -12 < 0
			\end{equation*}
			Így $f$-nek $P_3$-ban nincs lokális szélsőértéke.
			\item $P_4(-1,-2)$
			\begin{equation*}
				f''(-1,-2) =
				\begin{pmatrix}
					-12 & 0 \\
					0 & -12
				\end{pmatrix}, D_2 = 72 \cdot -12 \cdot -12 > 0 \text{ és } D_1 = -12 < 0 
			\end{equation*}
			Így $f$-nek $P_1$-ben lokális maximumhelye van.
		\end{enumerate}

\end{solution}

\end{document}