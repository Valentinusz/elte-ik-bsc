\documentclass[a4paper,12pt]{article}
\usepackage[utf8]{inputenc}
\usepackage{exsheets}

\DeclareInstance{exsheets-heading}{block-no-nr}{default}{
	attach = {
		main[l,vc]title[l,vc](0pt,0pt) ;
		main[r,vc]points[l,vc](\marginparsep,0pt)
	}
}

\RenewQuSolPair
{question}[headings=runin]
{solution}[headings=block-no-nr]

\SetupExSheets{
	counter-format=qu.,
	solution/print=true ,
	question/name=Feladat,
	solution/name=Megoldás.
}

\usepackage{tasks}
\usepackage[hungarian]{babel}
\usepackage{amsmath}
\usepackage{mathtools}
\usepackage{amsthm}
\usepackage[shortlabels]{enumitem}
\usepackage{amsfonts}
\usepackage{amssymb}
\usepackage{graphicx}
\usepackage{wrapfig}
\graphicspath{{./images/}}
\usepackage{float}
\usepackage{multicol}
\usepackage{tikz}
\usepackage{minted}
\usetikzlibrary{positioning,shapes,fit,arrows}
\tikzset{every picture/.style={line width=0.75pt}} %set default line width to 0.75pt    

\title{\huge{Diszkrét modellek alkalmazásai} \\ \large  I. zh - minta}
\author{kidolgozta: Boda Bálint}
\date{2022. őszi félév}

\DeclareMathOperator{\interior}{int}
\DeclareMathOperator{\lf}{lf}
\DeclareMathOperator{\lnko}{lnko}

\theoremstyle{definition}
\newtheorem{definition}{Definíció}
\newtheorem*{definition*}{Definíció}
\newtheorem*{remark}{Megjegyzés}
\newtheorem{theorem}{Tétel}
\newtheorem*{theorem*}{Tétel} 
\newtheorem*{example}{Példa}

\SetupExSheets{solution/print=true}
\SetupExSheets{question/name=}
\SetupExSheets{headings=runin}

\begin{document}
	\maketitle
	\begin{question}
		Bizonyítsuk be, hogy az $$ R = \{(a,b) | a \equiv b \pmod{10}\} \subseteq \mathbb{Z} \times \mathbb{Z} $$ reláció ekvivalenciareláció.
	\end{question}
	\begin{solution}
		Egy reláció ekvivalenciareláció, ha reflexív, szimmetrikus és tranzitív.
		\begin{enumerate}
			\item Reflexív: $\forall a \in \mathbb{Z}: a \equiv a \pmod{10} \iff a \pmod{10} = a \pmod{10} $ ami nyilván igaz 
			\item Szimmetrikus: $\forall a,b \in \mathbb{Z}: a \equiv b \pmod{10} \implies b \equiv a \pmod{10} \\ \iff a \pmod{10} = b \pmod{10} \implies b \pmod{10} = a \pmod{10} $, mivel az egyenlőség szimmetrikus reláció ezért a kongruencia is.
			\item Tranzitív: $$\forall a,b,c \in \mathbb{Z}: a \equiv b \pmod{10} \land b \equiv c \pmod {10} \implies a \equiv c \pmod{10} $$
			$$
			\iff a \pmod{10} = b \pmod{10} \land b \pmod{10} = c \pmod{10} $$ $$ \implies a \pmod{10} = c \pmod{10}, $$ mivel az egyenlőség tranzitív reláció ezért a kongruencia is.
		\end{enumerate}
	Az reláció 10 ekvivalenciaosztályra osztja $\mathbb{Z}$-t az alapján hogy az adott $a,b \in \mathbb{Z}$ osztályok 10-el vett osztási maradéka mennyi, melyek a 0,1,2,3,4,5,6,7,8,9 maradékosztályok.
	\end{solution}
	\newpage
	\begin{question}
		Oldja meg a $60x +16y = 60$ egyenletet az egész számok halmazán.
	\end{question}
	\begin{solution}
		$ 60x + 16y = 60 \iff 15x + 4y = 15 $
		\begin{alignat*}{2}
			\lnko{(15,4)} = 1 \quad 15 &= 3 \cdot 4 + 3 \implies 3 = 1 \cdot 15 - 3 \cdot 4 \\
			4 &= 1 \cdot 3 + 1 \implies 1 = 1 \cdot 4 - 1 \cdot 3 \implies 4 \cdot 4 - 1 \cdot 15 \\
			3 &= 3 \cdot 1 + 0
		\end{alignat*}
		$ a = -1,\; b = 4,\; c = 15, \; d=4 $ 
		\begin{alignat*}{2}
			15x + 4y &= 15 \\
			15a + 4b &= 1 \\
			15 \cdot -1 + 4 \cdot 4 &= 1 \quad \backslash \cdot 15 \\
			15 \cdot -15 + 4 \cdot 60 &= 15
		\end{alignat*}
		$ x_0 = -15, \; y_0=60 $
		\begin{alignat*}{2}
			x &= -15 + \frac{4}{1} \cdot t &= -15 + 4t &\qquad (t \in \mathbb{Z}) \\
			y &= 60 - \frac{15}{1} \cdot t &= 60 - 15t &\qquad (t \in \mathbb{Z}) 
		\end{alignat*}
	\end{solution}
	\begin{question}
		Határozza meg az Euklideszi algoritmussal a következő értékeket:
		\begin{tasks}
			\task $\lnko{(504, 150)}$
			\task $\lnko{(30, 22)}$
		\end{tasks}
	\end{question}
	\begin{solution}
		\begin{tasks}(2)
			\task{$\lnko{(504, 150)} = 6$ \begin{alignat*}{2}
				504 &= 3 \cdot 150 + 54 \\
				150 &= 2 \cdot 54 + 42 \\
				54 &= 1 \cdot 42 + 12 \\
				42 &= 3 \cdot 12 + 6 \\
				12 &= 2 \cdot 6 + 0
			\end{alignat*}}
			\task{ $\lnko{(30, 22)} = 2$ \begin{alignat*}{2}
				30 &= 1 \cdot 22 + 8 \\
				22 &= 2 \cdot 8 + 6 \\
				8 &= 1 \cdot 6 + 2 \\
				6 &= 3 \cdot 2 + 0
			\end{alignat*}}
		\end{tasks}
	\end{solution}
	\newpage
	\begin{question}
		Oldja meg a következő lineáris kongruenciákat:
		\begin{tasks}
			\task $ 16x \equiv 36 \pmod{28}$
			\task $ 15x \equiv 8 \pmod{20}$
		\end{tasks}
	\end{question}
	\begin{solution}
		\begin{tasks}
			\task { $ 16x \equiv 36 \pmod{28} $ 
				\begin{alignat*}
					116x \equiv 36 \pmod{28}
					&\iff 16x \equiv 64 \pmod{28} \\
					&\iff x \equiv 4 \pmod{7} \\
					&\implies x = 4 + 7t \quad (t \in \mathbb{Z})
				\end{alignat*}}
			\task { $ 15x \equiv 8 \pmod{20} \quad \lnko{(15,20)}=5 \qquad 5 \nmid 8 \implies \text{Nincs megoldás.} $ }
		\end{tasks}
	\end{solution}
	\begin{question}
		\begin{tasks}
			\task {Írjon függvényt, amely természetes számokat tartalmazó halmazt fogad paraméterként (üres halmaz esetén dobjon \mintinline{python3}|ValueError| kivételt). A függvény a számok valódi (nem triviális) osztóit állítsa elő úgy, hogy egy halmazzal tér vissza, amiben rendezett párok vannak: a pár első komponense az egyik természetes szám, a második komponense az első komponens valódi osztóinak halmaza. Hívja meg a függvényt példákkal (kapja el a dobott kivételt).}
			\task{Készítse el azt a listát, amelyben 112-nél nagyobb, 1000-nél kisebb prímszámok vannak, amelyek kongruensek 7-tel modulo 235.}
		\end{tasks}
	\end{question}
	\begin{question}
		Olvasson be a billentyűzetről egy m pozitív egész számot. Ábrázolja a következő irányított gráfot: csúcsai az $\{1; 2; ....;m\}$ összes 3-elemű részhalmazai; egy $\{a; b; c\}$ csúcsból akkor mutat irányított él egy $\{d; e; f\}$ csúcsba, ha $a + b + c < d * e * f$. Példaként rajzolja ki $m = 6$ esetben a gráfot. 		
	\end{question}
\end{document}