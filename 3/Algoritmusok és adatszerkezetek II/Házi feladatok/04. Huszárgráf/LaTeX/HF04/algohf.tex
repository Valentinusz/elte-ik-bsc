\documentclass[a4paper,12pt]{article}
\usepackage[margin=1in]{geometry}
\usepackage[utf8]{inputenc}

\usepackage[hungarian]{babel}
\usepackage{amsmath}
\usepackage{mathtools}
\usepackage{amsthm}
\usepackage{amsfonts}
\usepackage{amssymb}
\usepackage{graphicx}
\graphicspath{{./images/}}
\usepackage{stuki}
\usepackage{stukicommands}
\usepackage{float}
\usepackage{multicol}
\usepackage{animate}

\title{\huge{Algoritmusok és adatszerkezetek II} \\ \large Sakk}
\author{Boda Bálint}
\date{2022. őszi félév}

\begin{document}
    \maketitle
    \noindent
    A sakkban a huszár kétféleképpen tud mozogni: \begin{itemize}
    	\item függőlegesen két mezőt és vízszintesen egyet
    	\item függőlegesen egy mezőt és vízszintesen kettőt
    \end{itemize}
	Ha elkezdjük beszínezni a huszár által $n \; \left( n \in 0,1,2\dots\right)$ lépésből elérhető mezőket (melyet a következő animáció szemléltet),  
	
	\begin{center}
		\animategraphics[controls]{1}{h}{0}{4}
	\end{center}

	\noindent
	megfigyelhetjük, hogy a beszínezés eljárása sok tekintetben hasonlít a szélességi bejárás algoritmusára, gyakorlatilag annak egy olyan módosítása mely az algoritmussal egyidőben építi fel a gráfot.

	Könnyű meggondolni, hogy bizonyos táblaméreteknél a huszár nem tudja az összes mezőt elérni, illetve kellően kicsi tábla esetén még mozogni sem tud, azaz előfordulhat olyan eset, hogy nem találunk utat a keresett mezőbe.
	
	
	\newpage
	\noindent
	A csúcsok reprezentálására vezessük be a következő típust:
	\begin{center}
		\struct{Vertex}
		+ int $i$ \\
		+ int $j$ \\
		\hline
		+ Vertex(r,c) \{$i$ = $r$; $j$ = $c$\} \\
		\eoStruct
	\end{center}

	
	Készítsünk egy segédeljárást, mely megadja egy csúcsból a huszár által 1 lépéssel elérhető mezők halmazát.
	\begin{stuki*}[10cm]{getNeighbours($n,m:\mathbb{N}^{+},i,j:\mathbb{N}): Vertex \left\lbrace \right\rbrace $}
		\stm{V: Vertex \left\lbrace \right\rbrace \linecomment{\text{set of vertices}}}
		\stm{V \coloneq V \cup \left\lbrace Vertex(i-2,j-1), Vertex(i-2,j+1) \right\rbrace }
		\stm{V \coloneq V \cup \left\lbrace Vertex(i-1,j-2), Vertex(i-1,j+2) \right\rbrace }
		\stm{V \coloneq V \cup \left\lbrace Vertex(i+1,j-2), Vertex(i+1,j+2) \right\rbrace }
		\stm{V \coloneq V \cup \left\lbrace Vertex(i+2,j-1), Vertex(i+2,j+1) \right\rbrace }
		\begin{WHILE}{2}{\stm{\forall v \in V}}
			\begin{IF}{1}{\stm{v.i < 1 \; \lor \; v.i > n \; \lor \; v.j < 1 \; \lor \; v.j > m}}
				\stm{V \coloneq V \setminus \left\lbrace v \right\rbrace }
				\ELSE
					\stm*{SKIP}
			\end{IF}
		\end{WHILE}
		\stm*{\textbf{return} $V$}
	\end{stuki*}


    \begin{stuki*}[14cm]{shortestKnightPathLength($n,m:\mathbb{N}^{+},i_1,j_1,i_2,j_2:\mathbb{N}) : \mathbb{N}$}
    	 \begin{IF}{1}{\stm{i_1 = i_2 \; \land \; j_1 = j_2}}
    		\stm*{\textbf{return} 0}
    		\ELSE
    		\stm*{SKIP}
    	\end{IF}
    	\stm{d: \mathbb{N}[n][m] \linecomment{\text{distance of a given vertex from source}}}
    	\begin{WHILE}{2}{\stm{i \coloneq 1 .. n}}
    		\begin{WHILE}{1}{\stm{j \coloneq 1 .. m}}
    			\stm{d[i][j] \coloneq \infty}
    		\end{WHILE}
    	\end{WHILE}
        \stm{s \coloneq Vertex(i_1,j_1); \; d[s.i,s.j] = 0}

        \stm{Q: Queue; \; Q \mathop{.add}(s)}
        \begin{WHILE}{8}{\stm{\lnot Q\mathop{.isEmpty()}}}
        	\stm{u \coloneq Q\mathop{.rem()}}
        	\stm{neighbours \coloneq getNeighbours(n,m,u.i,u.j)}
        	\begin{WHILE}{5}{\stm{\forall v \in neighbours}}
        		\begin{IF}{4}{\stm{d(v.i,v.j) = \infty}}
        				\stm{d[v.i][v.j] = d[u.i][u.j] + 1}
        				\begin{IF}{1}{\stm{v.i = i_2 \; \land \; v.j = j_2}}
        					\stm*{\textbf{return} $d[v.i][v.j]$}
        				\ELSE
        					\stm*{SKIP}
        				\end{IF}
        				\stm{Q.add(v)}
        			\ELSE
        				\stm*{SKIP}
        		\end{IF}
        	\end{WHILE}
        \end{WHILE}
    	\stm*{\textbf{return} $\infty$}
    \end{stuki*}
\end{document}