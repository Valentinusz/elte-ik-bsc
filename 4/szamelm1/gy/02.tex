\section{Generatív grammatika}
\begin{definition*}[grammatika]
	Egy $ G = (N, T, P, S) $ rendezett négyest (generatív) grammatikának vagy nyelvtannak nevezünk, ha $N$ és $T$ diszjunkt (azaz $N \cap T = \emptyset $) véges ábécék. Ekkor 
	\begin{itemize}
		\item $N$ a nem terminális szimbólumok halmaza,
		\item $T$ (vagy $ \Sigma $) a terminális szimbólumok halmaza,
		\item $S \in N$ a grammatika kezdőszimbóluma,
		\item $P = \left\lbrace (x,y) \; | \; x,y \in \left( N \cup T \right)^*  \text{ szavak úgy, hogy } x \text{ legalább egy nem terminális betűt tartalmaz} \right\rbrace $, az ún. (átírási) szabályok (vagy produkciók) halmaza.
	\end{itemize}
\end{definition*}

\begin{notation*}
	Gyakran $(x,y)$ helyett az $x \rightarrow y$ jelölést használjuk, egy szabály leírására. Természetesen ez csak akkor lehetséges ha az adott ábécének nem eleme $\rightarrow$. 
\end{notation*}
\vspace{6pt}
\begin{definition*}[egylépéses levezetés]
	Legyen $ G = (N, T, P, S) $ egy grammatika és legyen $u, v \in (N \cup T)^*$. Azt mondjuk $v$ közvetlen levezethető az $u$ szóból $G$-ben (jelekkel: $ u \Rightarrow_G v $), ha
	\[
	\exists (x, y) \in P: \; u = u_1xu_2 \text{ és } v = u_1yu_2, \quad \left(u_1, u_2 \in (N \cup T)^*\right)
	\]  
\end{definition*}
\vspace{6pt}
\begin{definition*}[többlépéses levezetés]
	Legyen $ G = (N, T, P, S) $ egy grammatika és legyen $u, v \in (N \cup T)^*$. Azt mondjuk $v$ több lépésben levezethető az $u$ szóból $G$-ben (jelekkel: $ u \Rightarrow_G^* v $), ha
	\[
	u = v \lor \exists \left( n \ge 1 \land w_0,\dots,w_n \in (N \cup T)^* \right)  \text{, hogy } w_{i-1} \Rightarrow_G w_i \, (1 \le i \le n), \; w_0 = u \text{ és } w_n = v
	\]
\end{definition*}
\vspace{6pt}
\begin{definition*}[generált nyelv]
	Legyen $ G = (N, T, P, S) $ egy grammatika, ekkor a $G$ által generált nyelvnek nevezzük az $S$ kezdőszimbólumból több lépésben levezethető terminális szavak halmazát, azaz a
	\[
		L(G) = \left\lbrace u \in T^* \; | \; S \Rightarrow_G^* u \right\rbrace 
	\]
	nyelvet.
\end{definition*}

\begin{example}
	Legyen
	\begin{align*}
		G &= \left( \left\lbrace S, A, B \right\rbrace, \left\lbrace a,b \right\rbrace P, S \right) \\
		P &= \left\lbrace S \rightarrow B | bb, \; B \rightarrow aaA, \; A \rightarrow a | \varepsilon \ \right\rbrace 
	\end{align*}
	Adjuk meg $L(G)$-t!

	\begin{remark}
		Egy $ S \rightarrow B | bb $ az $ S \rightarrow B $ és $ S \rightarrow bb $ szabályokat jelöli.
	\end{remark}

	\begin{align*}
		S &\rightarrow bb \\
		S &\rightarrow B \rightarrow aaA \rightarrow a \\
		S &\rightarrow B \rightarrow aaA \rightarrow \varepsilon
	\end{align*}
	Így: $L(G) = \left\lbrace bb, aa, a \right\rbrace $. 
\end{example}

\subsection{Feladatok}

\begin{exercise}
	Legyen $ G_i = (\left\lbrace S, A, B \right\rbrace, \left\lbrace a,b \right\rbrace, P_i, S) $. Határozzuk meg az $L(G_i)$ nyelvet, ha
	\begin{itemize}
		\item $P_1 = \left\lbrace S \rightarrow aaS | a \right\rbrace $
		\item $P_2 = \left\lbrace S \rightarrow aSb | \varepsilon \right\rbrace $
		\item $P_3 = \left\lbrace S \rightarrow ASB | \varepsilon, \, AB \rightarrow BA, \, BA \rightarrow AB, \, A \rightarrow a, \, B \rightarrow b \right\rbrace $
	\end{itemize}
\end{exercise}
		
\begin{solution}
	\begin{align*}
		L(G_1) &= \left\lbrace a, aaa, aaaaa \dots \right\rbrace = \left\lbrace a^{(2n+1)} \; | \; n \ge 0 \right\rbrace \\
		L(G_2) &= \left\lbrace \varepsilon, ab, aabb \dots \right\rbrace = \left\lbrace a^nb^n \; | \; n \ge 0 \right\rbrace
	\end{align*}
	A harmadik nyelv meghatározása már nehezebb feladat. Tekintsünk pár példa levezetést:
	\begin{align*}
		S &\rightarrow A\underline{S}B \rightarrow \underline{A}B \rightarrow a\underline{B} \rightarrow ab \\
		S &\rightarrow ASB \rightarrow AB \rightarrow BA \rightarrow bA \rightarrow ba \\
		S &\rightarrow A\underline{S}B \rightarrow AA\underline{S}BB \rightarrow A\underline{AB}B \rightarrow \underline{AB}AB \rightarrow BA\underline{AB} \rightarrow \underline{B}ABA \rightarrow \dots \rightarrow baba \\
		S &\rightarrow ASB \rightarrow AASBB \rightarrow AABB \rightarrow ABAB \rightarrow BAAB \rightarrow baab
	\end{align*}
	Ezek alapján $L(G_3) = \left\lbrace u \in \left\lbrace a,b \right\rbrace^* \; | \; l_a(u) = l_b(u) \right\rbrace $.
\end{solution}

\section{A grammatikák Chomsky féle osztályzása}
Legyen $ G = (N, T, P, S) $ egy grammatika. A $G$ grammatika $i$-típusú $ (i = 0,1,2,3) $, ha a $P$ szabályhalmazra teljesülnek a következők:
\begin{itemize}
	\item $ i = 0 $ (mondatszerkezetű grammatika): nincs korlátozás
	\item { $ i = 1 $ (környezetfüggő grammatika):
		\begin{itemize}
			\item $P$ minden szabálya $u_1Au_2 \rightarrow u_1vu_2 $ alakú, ahol $u_1, u_2, v \in (N \cup T)^*, A \in N $ és $ v \ne \varepsilon $
			\item Kivétel: $P$ tartalmazhatja az $ S \rightarrow \varepsilon $ szabályt, de csak akkor, ha $S$ nem fordul elő egyetlen szabály jobb oldalán sem.
		\end{itemize}
	}
	\item { $ i = 2 $ (környezetfüggetlen): $ P $ minden szabálya $A \rightarrow v $ alakú ($A \in N, \; v \in (N \cup T)^*) $
	}
	\item { $ i = 3 $ (reguláris): $ P $ minden szabálya $A \rightarrow uB $ vagy $A \rightarrow u $ alakú ($A,B \in N, \; u \in T^*) $
	}
\end{itemize}
Az adott osztályokat $\mathcal{G}_i$-vel jelöljük.

\begin{definition*}[nyelvosztály]
	Az $i$ típusú nyelvek osztályának nevezzük a 
	\[
	\mathcal{L}_i = \left\lbrace L \; | \; \exists G \in \mathcal{G}_i \text{, hogy } L = L(G) \right\rbrace 
	\]
\end{definition*}

\begin{theorem}[Chomsky nyelvhierarchia tétel]
	\[
		\mathcal{L}_3 \subset \mathcal{L}_2 \subset \mathcal{L}_1 \subset \mathcal{L}_0
	\]
\end{theorem}

\subsection{Feladatok}
\setcounter{exercise}{0}
\begin{exercise}
	Írjuk fel azt a grammatikát, mely a 4-el osztható bináris számok nyelvét generálja! Milyen osztályba sorolható a generált nyelv?
\end{exercise}

\begin{solution}
	Egy kettes számrendszerbeli szám akkor osztható néggyel, ha utolsó két számjegye 0. Gondoskodnunk kell továbbá arról, hogy ne legyenek felesleges nullák az elején. Így
	\[
	G = \left( \left\lbrace S, B \right\rbrace, \, \left\lbrace 0,1, \varepsilon \right\rbrace, \, \left\lbrace  S \rightarrow \underbrace{0}_{3.} | \underbrace{1B00}_{2.}, \,  B \rightarrow \underbrace{\varepsilon}_3.|\underbrace{0B}_{3.}|\underbrace{1B}_{3.} \right\rbrace , \, S \right)
	\]
	(Az adott szabály jobb oldala alatt tüntettem fel annak szintjét.) Mivel kettes a legkisebb szint ezért a generált nyelv is 2-es szintű.
	\\[4pt]
	A feladat megoldható más módon is:
	\[
	G = \left( \left\lbrace S, A \right\rbrace, \, \left\lbrace 0,1, \varepsilon \right\rbrace, \, \left\lbrace  S \rightarrow 0 | 1A, \,  A \rightarrow \varepsilon|0A|1A \right\rbrace , \, S \right)
	\]
	amiből már reguláris nyelv adódik.
\end{solution}

\begin{exercise}
	Írjuk fel azt a grammatikát, ami az $ L(G) = \left\lbrace a^n b^m c^n \; | n \ge 0, \, m \ge 3 \right\rbrace  $, nyelvet generálja!
\end{exercise}
\begin{solution}
	Írjuk fel $L(G)$ néhány elemét: $ \left\lbrace bbb, abbbc, aabbbcc \dots \right\rbrace $. Így $ G = (\left\lbrace  S,A,B,C \right\rbrace , \left\lbrace a,b,c \right\rbrace, P, S) $, ahol
	\begin{align*}
		P = \left\lbrace \begin{array}{l}
			S \rightarrow ASC, \\
			S \rightarrow BBB, \\
			A \rightarrow a, \\
			B \rightarrow b, \\
			C \rightarrow c
		\end{array} \right\rbrace 
	\end{align*}
\end{solution}

\section{Reguláris kifejezések}