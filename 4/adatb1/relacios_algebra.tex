\documentclass[a4paper,12pt]{article}

\usepackage[margin=0.75in]{geometry}
\usepackage[hungarian]{babel}
\usepackage[utf8]{inputenc}
\usepackage[T1]{fontenc}
\usepackage{lmodern}


\usepackage{amsmath}
\usepackage{mathtools}
\usepackage{amsthm}
\usepackage{amsfonts}
\usepackage{amssymb}
\usepackage{graphicx}
\usepackage{wrapfig}
\graphicspath{{./images/}}
\usepackage{float}
\usepackage{multicol}
\usepackage{xsim}
\usepackage{tasks}

\theoremstyle{definition}
\newtheorem{definition}{Definíció}
\newtheorem*{definition*}{Definíció}
\newtheorem*{remark}{Megjegyzés}
\newtheorem{theorem}{Tétel}
\newtheorem*{theorem*}{Tétel}
\newtheorem*{example}{Példa}
\newtheorem{notation}{Jelölés}
\newtheorem*{notation*}{Jelölés}

\DeclareExerciseTranslation{magyar}{exercise}{feladat}
\DeclareExerciseEnvironmentTemplate{feladat}{
	{%
		\par\vspace{\baselineskip}
		\noindent
		\textbf{\GetExerciseProperty{counter}.~ \XSIMmixedcase{\GetExerciseName}}%
		\IfInsideSolutionF
		{
			\GetExercisePropertyT{subtitle}%
			{ {\normalfont\itshape\PropertyValue}}%
		}
	}
}
{}

\DeclareExerciseEnvironmentTemplate{megoldas}{
	{%
		\par\vspace{\baselineskip}
		\noindent
		\textbf{\XSIMmixedcase{\GetExerciseName}}%
		\IfInsideSolutionF
		{
			\GetExercisePropertyT{subtitle}%
			{ {\normalfont\itshape\PropertyValue}}%
		}
	}
}
{}
\xsimsetup{
	exercise/name=\XSIMtranslate{exercise},
	exercise/within=section,
	exercise/template=feladat,
	exercise/the-counter=\arabic{exercise},
	solution/name=megoldás,
	solution/print,
	solution/template=megoldas,
}

\usepackage{titling}
\setlength{\droptitle}{-2cm}

\title{\large{Adatbázisok I.} \\[-4pt] \huge{Relációs algebra} \\[-12pt]  \vspace{-15pt}}
\author{Boda Bálint}
\date{\vspace{-12pt}2023. tavaszi félév}



\begin{document}
\maketitle

\section{Alapműveletek}
\begin{itemize}
	\item {
		$ \pi_{A,B}{\left( R \right) } $ \textbf{projekció}, egy olyan relációt ad vissza, mely $R$ adott attribútumainak levetítéséből áll
	}
	\item {
		$ \sigma_{P}{\left( R \right)} $ \textbf{szelekció}, egy olyan relációt ad vissza, mely $R$ azon sorait tartalmazza, melyek megfelelnek a $P$ predikátumnak
	}
	\item {
		$ \rho_{(S_{(C,B)})}{\left( R \right)} $ \textbf{átnevezés}, egy olyan $S$ relációt ad vissza, mely sorai megegyeznek $R$ soraival, de attribútumai pedig rendre $ C,B $
	}
	\item {
		halmazműveletek ($ \cup $, $ \cap $, $ \setminus $)
	}
	\item {
		Descartes-szorzat ($ \times $, [SQL: CROSS JOIN])
	}
	\item {
	Természetes összekapcsolás ($ \bowtie $, [SQL: NATURAL JOIN]), egy olyan relációt add vissza, mely olyan sorpárokból áll, ahol $ R $ és $ S $ azonos attribútumain megegyeznek.
	}

	\item {
		Théta-összekapcsolás ($ \bowtie_\theta $, [SQL: JOIN és ON]), $ R \bowtie_P S = \sigma_P{\left( R \times S \right) } $, ahol $P$ egy feltétel.
	}
\end{itemize}
\section{Feladatok}
Tekintsük a $ szeret $ táblát:
\begin{table}[H]
	\centering
	\begin{tabular}{|c|c|}
		\hline
		nev & gyumolcs \\
		\hline
		Micimacko & Alma \\
		\hline
		Micimacko & Korte \\
		\hline
		Tigris & Alma \\
		\hline
		Bagoly & Eper \\
		\hline
	\end{tabular}
\end{table}
	
\begin{exercise}
	Adjuk meg azokat a gyümölcsöket melyeket Micimackó szeret!
	
	\[
	\pi_{gyumolcs}{\left(  \sigma_{nev = Micimacko}{\left( szeret  \right) }\right) }
	\]
\end{exercise}

\begin{exercise}
Adjuk meg azokat a gyümölcsöket melyeket Micimackó nem szeret!
	\[
	\pi_{gyumolcs}{\left( szeret \right)} \setminus \pi_{gy\ddot{u}umolcs}{\left(  \sigma_{nev = Micimacko}{\left( szeret  \right) }\right) }
	\]
\end{exercise}

\begin{exercise}
	Kik szeretik az almát?
	\[
	\pi_{nev}(\sigma_{gyumolcs = Alma}(szeret))
	\]
\end{exercise}
\newpage
\begin{exercise}
	\begin{tasks}
		\task{Kik nem szeretik a körtét?
			\[
			\pi_{nev}(szeret \setminus (\sigma_{gyumolcs = Korte}(szeret))
			\]
		}
		\task{Kik nem szeretik a körtét, de valami mást igen?
			\[
			\pi_{nev}(\sigma_{gyumolcs \ne Körte}(szeret))
			\]
		}
	\end{tasks}
\end{exercise}

\begin{exercise}
	Kik szeretik vagy a dinnyét vagy a körtét?
	\[
	\pi_{nev}(\sigma_{gyumolcs = Korte \, \lor \, gyumolcs = Dinnye}(szeret))
	\]
\end{exercise}

\begin{exercise}
	Kik szeretik az almát is és a körtét is?
	\[
	\pi_{nev}(\sigma_{gyumolcs = Alma}(szeret)) \cap \pi_{nev}(\sigma_{gyumolcs = Korte}(szeret))
	\]
\end{exercise}

\begin{exercise}
	Kik azok, akik szeretik az almát, de nem szeretik a körtét?
	\[
	\pi_{nev}(\sigma_{gyumolcs = Alma}(szeret)) \setminus \pi_{nev}(\sigma_{gyumolcs = Korte}(szeret))
	\]
\end{exercise}

\begin{exercise}
	Kik szeretnek legalább kétféle gyümölcsöt?
	\[
	\pi_{sz1.nev}\left(\sigma_{sz1.nev = sz2.nev \, \land \, sz1.gyumolcs \ne sz2.gyumolcs}{ \left( \rho_{sz1}\left(szeret \right) \times \rho_{sz2} \left( szeret \right) \right) }\right)
	\]
\end{exercise}

\begin{exercise}
	Kik szeretnek legalább háromféle gyümölcsöt?
	\begin{align*}
	& S = \rho_{sz1}\left(szeret\right) \times \rho_{sz2}\left(szeret\right) \times \rho_{sz3}\left(szeret\right) \\
	&\pi_{sz1.nev}\left(\sigma_{sz1.nev = sz2.nev \, \land \, sz2.nev = sz3.nev \, \land \, sz1.gyumolcs \ne sz2.gyumolcs \, \land \, sz2.gyumolcs \ne sz3.gyumolcs}\right){\left( S \right) }
	\end{align*}
\end{exercise}

\begin{exercise}
	Kik szeretnek legfeljebb kétféle gyümölcsöt.
	\[
	\pi_{név}\left(szeret \right) \setminus {\text{9. feladat}}
	\]
\end{exercise}

\begin{exercise}
	Kik szeretnek pontosan kétféle gyümölcsöt?
	\[
	\text{8. feladat} \setminus \text{9. feladat}
	\]
\end{exercise}

\end{document}