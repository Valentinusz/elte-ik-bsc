\documentclass[a4paper,12pt]{article}

\usepackage[margin=0.75in,showframe]{geometry}
\usepackage[hungarian]{babel}
\usepackage[utf8]{inputenc}
\usepackage{exsheets}

\DeclareInstance{exsheets-heading}{block-no-nr}{default}{
    attach = {
        main[l,vc]title[l,vc](0pt,0pt) ;
        main[r,vc]points[l,vc](\marginparsep,0pt)
    }
}
\RenewQuSolPair
{question}[headings=runin]
{solution}[headings=block-no-nr]

\SetupExSheets{
    counter-format=qu.,
    solution/print=true ,
    question/name=Feladat,
    solution/name=Megoldás.
}

\SetupExSheets{solution/print=true}
\SetupExSheets{question/name=}
\SetupExSheets{headings=runin}

\usepackage{tasks}
\usepackage{amsmath}
\usepackage{mathtools}
\usepackage{amsthm}
\usepackage[shortlabels]{enumitem}
\usepackage{amsfonts}
\usepackage{amssymb}
\usepackage{graphicx}
\graphicspath{{./images/}}
\usepackage{float}
\usepackage{multicol}
\usepackage{mathtools}

\title{\huge{Analízis II} \\ \large 10. Házi feladat}
\author{Boda Bálint}
\date{2022. őszi félév}

\theoremstyle{definition}
\newtheorem*{remark}{Megjegyzés}

\DeclareMathOperator{\tg}{tg}
\DeclareMathOperator{\arctg}{arctg}



\begin{document}
	\maketitle
	\begin{question}
		Bizonyítsa be, hogy:
		\begin{tasks}
			\task{ Az $f$ függvény folytonos a $(0,0)$ pontban!
			\begin{equation*}
				f(x,y) \coloneq \begin{cases}
					\displaystyle \frac{x^3 y^2}{3x^2+2y^2}, &\text{ha } (x,y) \in \mathbb{R}^2 \setminus \{ (0,0) \} \\
					0, &\text{ha } (x,y) = (0,0)
				\end{cases}
			\end{equation*}
			A folytonosság definíciója alapján azt kell megmutatni, hogy:
			\begin{equation}
				\forall \varepsilon > 0: \exists \delta > 0: \forall x \in D_f, || (x,y) - (0,0) || < \delta: | f(x,y) - f(0,0) | < \varepsilon
			\end{equation}
			Rögzítsünk egy $\varepsilon > 0$ valós számot. Ha $(x,y) = (0,0)$, akkor ${|f(x,y)-f(0,0)| = 0 < \varepsilon}$
			\\[4pt]
			Ha $(x,y) \in \mathbb{R}^2 \setminus \{(0,0)\} $, akkor
			\begin{align*}
				\left| f(x,y) - f(0,0) \right| &= \left| \frac{x^3 y^2}{3x^2+2y^2} - 0 \right| = \frac{|x^3|y^2}{3x^2+2y^2} = \frac{y^2}{3x^2+2y^2} \cdot |x^3| \le \frac{3x^2+2y^2}{3x^2+2y^2} \cdot |x^3| \\
				&\le (\text{tfh. } \underline{ || (x,y) || < 1 } \text{, ekkor } |y| < 1) \le |y^2| \le x^2 + y^2 = \underbrace{ \bigl|\bigl|(x,y)\bigr|\bigr|^2 < \varepsilon}_{||(x,y)|| < \sqrt{\varepsilon}}
			\end{align*} 
			Így, ha $ \delta \coloneq \min{\{\underline{1}, \sqrt{\varepsilon}\}} $, akkor $(1)$ teljesül, ami azt jelenti, hogy $f \in C \bigl \lbrace (0,0) \bigr \rbrace $.
 			}
			\task A $g$ függvénynek nincs határértéke a $(0,0)$ pontban!
				\[
				g(x,y) \coloneq \frac{x^2 y^2}{x^2 y^2 + (x-y)^2} \quad \bigl( (x,y) \in \mathbb{R}^2 \setminus \{(0,0)\} \bigr)
				\]
			A határértékekre vonatkozó átviteli elv alapján két olyan $(0,0)$-hoz tartó sorozatot kell találni, melyekre a függvényértékek sorozatának határértéke különböző.
			
			Rögzített $m \in \mathbb{R}$ esetén tekintsük $g$ értékeit az $y = mx$ egyenletű egyenes pontjaiban:
			\begin{align*}
				g(x,y)=g(x,mx)&=\frac{x^2 \cdot m^2x^2}{x^2 \cdot m^2x^2 + (x - mx)^2} = \frac{m^2x^4}{m^2x^4 + (x-mx)^2} \\
				&= \frac{m^2x^4}{m^2x^4+x^2(1-m)^2} = \frac{m^2}{m^2 + x^{-2}(1-m)^2}
			\end{align*}
			Ekkor
			\begin{itemize}
				\item ha $m=0$ és így $(x_n,y_n) \coloneq \left( \frac{1}{n},0 \right) \rightarrow (0,0) \implies g(x_n,y_n) = 0$
				\item ha $m=1$ és így $(u_n,v_n) \coloneq \left( \frac{1}{n}, \frac{1}{n} \right) \rightarrow (0,0) \implies g(u_n,v_n) = \frac{1}{1} = 1$ 
			\end{itemize}
			Mivel
			\[ \lim\limits_{n \leftarrow +\infty}{(x_n,y_n)}
			= (0,0) =
			\lim\limits_{n \leftarrow +\infty}{(u_n,v_n)} \]
			de
			\[ \lim\limits_{n \leftarrow +\infty}{g(x_n,y_n)}
			= 0 \ne 1 =
			\lim\limits_{n \leftarrow +\infty}{g(u_n,v_n)} \]
			ezért a $g$ függvénynek nincs határértéke $(0,0)$ pontban.
		\end{tasks}
	\end{question}
	\newpage
	\begin{question}
		Számolja ki az
		\[
		f(x,y) \coloneq xe^{yx}-xy \quad \bigl( (x,y) \in \mathbb{R}^2 \bigr)
		\]
		függvény iránymenti deriváltját az $(1,1)$ pontban a $v = (3,4)$ vektor által meghatározott irány mentén!
		\\[4pt]
		Mivel $v$ nem egységvektor ezért elő kell állítanunk a normáját.
		\[
		u = \frac{v}{||v||} = \begin{pmatrix} 3 \\ 4 \end{pmatrix} \cdot \frac{1}{5} = \begin{pmatrix} 0,6 \\ 0,8 \end{pmatrix}
		\]
		Az iránymenti deriválhatósághoz azt kell megmutatni, hogy a
		\begin{align*}
			F_u(t) &\coloneq f(a + tu) = f(a_1 + tu_1, a_2 + tu_2) = f(1 + 0,6t; 1 + 0,8t) \\
			&= (1 + 0,6t)e^{(1+0,6t)(1+0,8t)} - (1+0,6t)(1+0,8t) \\
			&= (1 + 0,6t)e^{1+1,4t+0,48t^2} - (1+1,4t+0,48t^2) \\
			&= (1 + 0,6t)e^{1+1,4t+0,48t^2} - 1 - 1,4t - 0,48t^2 \quad (t \in \mathbb{R})
		\end{align*}
		függvény deriválható a 0 pontban.
		Ez nyilván teljesül, így a derivált:
		\begin{align*}
			0,6 \cdot (e^{1+1,4t+0,48t^2}) + (1 + 0,6t)e^{1+1,4t+0,48t^2} \cdot (1,4 + 0.96t) - 1,4 -0,96t
		\end{align*}
		 és $F_u'(0) = 0,6e + 1,4e - 1,4 = 2e - 1,4 $. Ezért $f$-nek létezik $v$ irányú iránymenti deriváltja az (1,1) pontban és értéke $2e - 1,4$.
		 Ellenőrizzük a megoldás helyességét a parciális deriváltak és az iránymenti deriváltak kapcsolatára vonatkozó tétel segítéségével:
		\begin{align*}
			\partial_1f(x,y) &= e^{yx} + xe^{yx}y - y \\
			\partial_2f(x,y) &= x^2e^{yx} - x
		\end{align*}
		A tétel alapján:
		\begin{align*}
			\partial_uf(1,1) &= \partial_1f(1,1) \cdot u_1 + \partial_2f(1,1) \cdot u_2 =  ( e^{yx} + xe^{yx}y - y ) \cdot 0.6 + ( x^2e^{yx} - x ) \cdot 0.8 \\
			&= 0.6 \cdot ( e^{1 \cdot 1} + 1 \cdot e^{1 \cdot 1} \cdot 1 - 1) + 0.4 \cdot (1^2e^{1 \cdot 1} - 1) = 0.6(2e-1)+0.8(e-1) \\
			&=  \frac{12e-6}{10} + \frac{8e-8}{10} = 2e -1,4
		\end{align*}
	\end{question}
\end{document}