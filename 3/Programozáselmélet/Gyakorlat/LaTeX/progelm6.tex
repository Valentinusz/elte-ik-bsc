%! Author = Val
%! Date = 2022. 10. 20.

%! Author = Valentinusz
%! Date = 2022. 10. 19.

\documentclass[a4paper,12pt]{article}

\usepackage[margin=1in]{geometry}

\usepackage[utf8]{inputenc}
\usepackage{exsheets}
\usepackage{centernot}
\usepackage{listings}

\DeclareInstance{exsheets-heading}{block-no-nr}{default}{
    attach = {
        main[l,vc]title[l,vc](0pt,0pt) ;
        main[r,vc]points[l,vc](\marginparsep,0pt)
    }
}

\RenewQuSolPair
{question}[headings=runin]
{solution}[headings=block-no-nr]

\SetupExSheets{
    counter-format=qu.,
    solution/print=true ,
    question/name=Feladat,
    solution/name=Megoldás.
}

\usepackage{tasks}
\usepackage[hungarian]{babel}
\usepackage{amsmath}
\usepackage{mathtools}
\usepackage{amsthm}
\usepackage[shortlabels]{enumitem}
\usepackage{amsfonts}
\usepackage{amssymb}
\usepackage{graphicx}
\usepackage{wrapfig}
\graphicspath{{./images/}}
\usepackage{float}
\usepackage{multicol}
\usepackage{tikz}
\usepackage{booktabs}
\usepackage{lstmisc}
\usetikzlibrary{positioning,shapes,fit,arrows}
\tikzset{every picture/.style={line width=0.75pt}} %set default line width to 0.75pt

\title{\huge{Programozáselmélet} \\ \large 6. gyakorlat}
\author{Boda Bálint}
\date{2022. őszi félév}

\theoremstyle{definition}
\newtheorem{definition}{Definíció}
\newtheorem*{definition*}{Definíció}
\newtheorem*{remark}{Megjegyzés}
\newtheorem{theorem}{Tétel}
\newtheorem*{theorem*}{Tétel}
\newtheorem*{example}{Példa}

\DeclareMathOperator{\lf}{lf}
\DeclareMathOperator{\ps}{p(S)}
\DeclareMathOperator{\prim}{prím}

\SetupExSheets{solution/print=true}
\SetupExSheets{question/name=}
\SetupExSheets{headings=runin}

\begin{document}
    \maketitle
    \begin{question}
        Döntsük el egy adott pozitív egész számról, hogy prím-e.
    \end{question}
    \begin{solution}
        A paraméterteret úgy érdemes megválasztani, hogy az állapottér egy olyan altere legyen, melyben csak olyan változók vannak, melyek befogylásolják a végeredményt.
        \begin{align*}
            A &= (x: \mathbb{N}^+, l: \mathbb{L}) \\
            B &= (x': \mathbb{N}^+) \\
            Q &= (x * 2 = x') \\
            R &= (Q \; \land \; l = \prim(x))
        \end{align*}
        \[ \prim{}: \mathbb{N} \rightarrow \mathbb{L} \]
        \[ \prim(x) \coloneqq
        \begin{cases}
            hamis, & x = 1 \\
            \forall k \in \left[2..x-1\right]: k \centernot | x, & x \ne 1
        \end{cases} \]
        Az előfeltételben pedig érdemes kikötni azt, hogy azok a változók melyek megváltozása nem szükséges a feladathoz ne változhassanak meg.
        \begin{align*}
            A &= (x: \mathbb{N}^+, l: \mathbb{L}) \\
            B &= (x': \mathbb{N}^+) \\
            Q &= (x = x') \\
            R &= (Q \; \land \; l = \prim(x))
        \end{align*}
    \end{solution}

    \begin{question}
        Adott egy egészeket tartalmazó tömb.
        Határozzuk meg a legnagyobb elemét!
    \end{question}
    \begin{solution}
        \begin{align*}
            A &= (t: \mathbb{Z}^n, max: \mathbb{Z}) \\
            B &= (t': \mathbb{Z}^n) \\
            Q &= (t = t' \land n \ne 0) \\
            R &= (Q \; \land \; \forall i \in \left[ 1..n \right]: max \ge t\left[ i \right] \; \land \; \exists j \in \left[ 1..n \right]: max = t\left[ j \right])
        \end{align*}
    \end{solution}

    \begin{question}
        Adott egy egészeket tartalmazó tömb.
        Ha tartalmaz pozitív elemeket, akkor keressük meg a legnagyobb elemét, különben a legkisebbet.
    \end{question}
    \begin{solution}
        \begin{align*}
            A &= (x: \mathbb{Z}^n, ext:\mathbb{Z}) \\
            B &= (x': \mathbb{Z}^n) \\
            Q &= (t = t' \land n \ne 0) \\
            R &= \left(Q \; \land \; l =
                \begin{cases}
                    \forall i \in \left[ 1..n \right]: ext \ge t\left[ i \right] \; \land \; \exists j \in \left[ 1..n \right]: ext = t\left[ j \right], & \exists k \in \left[ 1..n \right]: t\left[ k \right ] > 0 \\
                    \forall i \in \left[ 1..n \right]: ext \le t\left[ i \right] \; \land \; \exists j \in \left[ 1..n \right]: ext = t\left[ j \right], & \exists k \in \left[ 1..n \right]: t\left[ k \right ] \le 0
                \end{cases} \right)
        \end{align*}
    \end{solution}

    \begin{question}
        Adott egy egészeket tartalmazó tömb.
        A tömb elemei egyediek.
        Rendezzük növekvően a tömböt!
    \end{question}
    \begin{solution}
        \begin{align*}
            A &= (t: \mathbb{Z}^n) \\
            B &= (t': \mathbb{Z}^n) \\
            Q &= (t' = t \; \land \; \forall i,j \in [1..n]: i \ne j \implies t[i] \ne t[j]) \\
            R &= \left(\forall i \in \left[ 1..n-1 \right]: t\left[ i \right]  \ge t\left[ i+1 \right] \; \land \; \left( \bigcup_{i=1}^n{\{t[i]\}} = \bigcup_{i=1}^n{\{t'[i]\}}  \right) \right)
        \end{align*}
    \end{solution}
\end{document}