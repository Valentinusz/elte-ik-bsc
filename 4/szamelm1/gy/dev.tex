\documentclass[a4paper,12pt]{article}

\usepackage[margin=0.75in]{geometry}
\usepackage[utf8]{inputenc}
\usepackage{t1enc}
\usepackage{lmodern}

\usepackage[table]{xcolor}
\usepackage{animate}

\usepackage{xsim}
\usepackage{tasks}

\DeclareExerciseTranslation{magyar}{exercise}{feladat}
\DeclareExerciseEnvironmentTemplate{feladat}{
	{%
		\par\vspace{\baselineskip}
		\noindent
		\textbf{\GetExerciseProperty{counter}.~ \XSIMmixedcase{\GetExerciseName}}%
		\IfInsideSolutionF
		{
			\GetExercisePropertyT{subtitle}%
			{ {\normalfont\itshape\PropertyValue}}%
		}
	}
}
{}

\DeclareExerciseEnvironmentTemplate{megoldas}{
	{%
		\par\vspace{\baselineskip}
		\noindent
		\textbf{\XSIMmixedcase{\GetExerciseName}}%
		\IfInsideSolutionF
		{
			\GetExercisePropertyT{subtitle}%
			{ {\normalfont\itshape\PropertyValue}}%
		}
	}
}
{}
\xsimsetup{
	exercise/name=\XSIMtranslate{exercise},
	exercise/within=section,
	exercise/template=feladat,
	exercise/the-counter=\arabic{exercise},
	solution/name=megoldás,
	solution/print,
	solution/template=megoldas,
}

\usepackage[hungarian]{babel}
\usepackage{amsmath}
\usepackage{mathtools}
\usepackage{amsthm}
\usepackage{amsfonts}
\usepackage{amssymb}
\usepackage{graphicx}
\usepackage{wrapfig}
\graphicspath{{./images/}}
\usepackage{float}
\usepackage{multicol}

\theoremstyle{definition}
\newtheorem{definition}{Definíció}
\newtheorem*{definition*}{Definíció}
\newtheorem*{remark}{Megjegyzés}
\newtheorem{theorem}{Tétel}
\newtheorem*{theorem*}{Tétel}
\newtheorem*{example}{Példa}
\newtheorem{notation}{Jelölés}
\newtheorem*{notation*}{Jelölés}

\usepackage{tikz}
\usetikzlibrary{automata,positioning}

\DeclareMathOperator{\tr}{\delta}

\begin{document}
\section{Reguláris kifejezések}

\subsection{A 3-as normálforma}
\begin{definition*}
	Egy $ G = (N, \Sigma, P, S) $ reguláris grammatikát hármas normálformájúnak (3NF) nevezünk, ha
	\[
	\forall p \in P \text{ produkciós szabályra igaz, hogy } A \rightarrow aB \text{ vagy } A \rightarrow \varepsilon \text{ alakú} \; \left( A,B \in N, a \in \Sigma \right) 
	\]
\end{definition*}
\paragraph{3NF-re hozás}
\begin{enumerate}
	\item Ellenőrzés, hogy a nyelv valóban reguláris-e
	\item Hossz redukció ($ A \rightarrow a_1,\dots,a_nB $ alakú szabályok felbontása)
	\item Láncmentesítés
	\begin{enumerate}
		\item Láncszabályok ($ A \rightarrow B $ alakúak) felírása
		\item $U$ halmazok meghatározása
		\item szabályhalmaz átalakítása
	\end{enumerate}
\end{enumerate}
\begin{example}
	Tekintsük a következő szabálykészletet:
	\[
	S \rightarrow abS \mid B, \quad B \rightarrow bB \mid V, \quad V \rightarrow aa \mid b
	\]
	\begin{enumerate}
		\item Hossz redukció:
		\begin{align*}
			S \rightarrow abS &\Rightarrow S \rightarrow aZ_1, \, Z_1 \rightarrow bS \\
			V \rightarrow aa &\Rightarrow V \rightarrow aZ_2, \, Z_2 \rightarrow aZ_3, \ Z_3 \rightarrow \varepsilon 
		\end{align*}
		Így az új szabálykészlet:
		\[
		S \rightarrow aZ_1 \mid B, \;
		Z_1 \rightarrow bS, \quad
		B \rightarrow bB \mid V, \quad
		V \rightarrow aZ_2, \;
		Z_2 \rightarrow aZ_3, \;
		Z_3 \rightarrow \varepsilon, \;
		V \rightarrow bZ_3
		\]
		\item Láncmentesítés: \\
		Láncszabályok: $ S \rightarrow B, \; B \rightarrow V $ \\
		$U$ halmazok:
		\begin{enumerate}
			\item $B$:
			\begin{align*}
				U_1(B) &= \left\lbrace B \right\rbrace \\
				U_2(B) &= U_1(B) \cup \left\lbrace \text{nemterminálisok melyekből } U_1(B) \text{ egyik eleme elérhető} \right\rbrace \\
				&= U_1(B) \cup \left\lbrace S \right\rbrace = \left\lbrace B, S \right\rbrace \\
				U_3(B) &= U_2(B) \cup \varnothing = \underline{\left\lbrace B, S \right\rbrace = U(B)}
			\end{align*}
			\item $V$:
			\begin{align*}
				U_1(V) &= \left\lbrace V \right\rbrace \\
				U_2(V) &= U_1(V) \cup \left\lbrace B \right\rbrace \\
				U_3(V) &= U_2(V) \cup \left\lbrace S \right\rbrace \\
				U_4(B) &= U_3(B) \cup \varnothing = \underline{\left\lbrace B, S, V \right\rbrace = U(V)}
			\end{align*}
			
		\end{enumerate}
		Szabályhalmaz átírása:
		\[
		S \rightarrow aZ_1 \mid bB \mid aZ_2 \mid bZ_3,  \;
		Z_1 \rightarrow bS, \quad
		B \rightarrow bB \mid aZ_2 \mid bZ_3 \quad
		Z_2 \rightarrow aZ_3, \;
		Z_3 \rightarrow \varepsilon
		\]
	\end{enumerate}
\end{example}
\end{document}