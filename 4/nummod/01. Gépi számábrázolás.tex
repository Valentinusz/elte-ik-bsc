\documentclass[a4paper,12pt]{article}

\usepackage[margin=0.75in]{geometry}
\usepackage[utf8]{inputenc}
\usepackage{t1enc}
\usepackage{lmodern}
\usepackage{exsheets}

\DeclareInstance{exsheets-heading}{block-no-nr}{default}{
	attach = {
		main[l,vc]title[l,vc](0pt,0pt) ;
		main[r,vc]points[l,vc](\marginparsep,0pt)
	}
}
\RenewQuSolPair
{question}[headings=runin]
{solution}[headings=block-no-nr]

\SetupExSheets{
	counter-format=qu.,
	solution/print=true ,
	question/name=Feladat,
	solution/name=Megoldás.
}

\usepackage[hungarian]{babel}
\usepackage{amsmath}
\usepackage{mathtools}
\usepackage{amsthm}
\usepackage{amsfonts}
\usepackage{amssymb}
\usepackage{graphicx}
\usepackage{wrapfig}
\graphicspath{{./images/}}
\usepackage{float}
\usepackage{multicol}
\usepackage{icomma}

\usepackage{titling}
\setlength{\droptitle}{-2cm}

\title{\huge{Numerikus módszerek \\[-4pt] A lebegőpontos számábrázolás egy modellje} \\ \large 1. gyakorlat \vspace{-15pt}}
\author{Boda Bálint}
\date{\vspace{-12pt}2023. tavaszi félév}

\DeclareMathOperator{\lf}{lf}

\theoremstyle{definition}
\newtheorem{definition}{Definíció}
\newtheorem*{definition*}{Definíció}
\newtheorem*{remark}{Megjegyzés}
\newtheorem{theorem}{Tétel}
\newtheorem*{theorem*}{Tétel}
\newtheorem*{example}{Példa}
\newtheorem{notation}{Jelölés}
\newtheorem*{notation*}{Jelölés}

\SetupExSheets{solution/print=true}
\SetupExSheets{question/name=}
\SetupExSheets{headings=runin}

\begin{document}
	\maketitle
	\section{Gépi számhalmaz}
	\vspace{-10pt}
		Alapötlet: tároljuk el a számokat ún. normalizált alakban. Például: $ 324 \rightsquigarrow 0,324 \cdot 10^3 $ Kettes számrendszerben: $ +0,101000100 \cdot 2^9 $.
	\begin{definition*}[Normalizált lebegőpontos szám]
		Legyen $ t \in \mathbb{N} $ (bitek száma) és $ m = \sum\limits_{i=1}^{t}{m_i \cdot 2^{i-1}} $, ahol $ m_1 = 1, \; m_i \in \left\lbrace 0, 1 \right\rbrace $. Ekkor normalizált lebegőpontos számnak nevezzük az
		\[
		a = \pm m \cdot 2^k \; (k \in \mathbb{Z})
		\]
		alakú számokat. Az $m$ számot az a szám mantisszájának a $k$ számot az a szám karakterisztikájának nevezzük.
	\end{definition*}

	\begin{notation*}
		Egy normalizált lebegőpontos számot általában a következő módon jelölünk:
		\[ a=\pm \left[ m_1 \dots m_t \, | \, k \right] \]
	\end{notation*}

	\begin{definition*}
		Legyen $ k^-, k^+ \in \mathbb{Z} $ és $ t \in \mathbb{N} $, ekkor gépi számhalmaznak nevezzük az
		\[
		M(t, k^-, k^+) = \left\lbrace a \text{ normalizált lebegőpontos szám, úgy, hogy } k^- \le k \le k^+ \right\rbrace \cup \left\lbrace 0 \right\rbrace 
		\]
		halmazt. Gyakorlatban hozzávesszük: $ \infty, -\infty $, NaN.
	\end{definition*}
	\subsection{Tulajdonságok}
	\begin{itemize}
		\item $ \frac{1}{2} \le m < 1 $
		\item $M$ szimmetrikus a $0$-ra
	\end{itemize}
	
	\section{Gépi számhalmaz nevezetes értékei}
	\subsection{Legkisebb pozitív szám}
	Csak $m_1 = 1$ az összes többi jegy nulla, a legkisebb karakterisztikát véve:
	\[
	\varepsilon_0 = \left[100 \dots 0 \, | \, k^-\right] = \frac{1}{2} \cdot 2^{k^-} = 2^{k^{-}-1}
	\]
	
	\subsection{Legnagyobb elem}
	Minden számjegy $1$ a legnagyobb karakterisztikával:
	\[
	M_{\infty} = \left[111 \dots 1 \, | \, k^+\right] = 1,00\dots00 \cdot 2^{k^+} - 0,00\dots01 \cdot 2^{k^+} = (1-2^{-t}) \cdot 2^{k^+}
	\]
	
	\subsection{Számosság}
	\[
	\left| M \right| = \underbrace{2}_{\text{előjelbit}} \cdot \underbrace{2^{t-1}}_{\text{lehetséges mantisszák száma}} \cdot \underbrace{(k^+ - k^- + 1)}_{\text{lehetséges karakterisztikák}} + \underbrace{1}_{0}
	\]
	
	\subsection{Egy és a rákövetkező szám különbsége}
	\[
	\varepsilon_1 = \left[ 100\dots01 \, | \, 1 \right] - \left[ 100\dots00 \, | \, 1 \right] = 2^{1-t}
	\]
\end{document}