\documentclass[a4paper,12pt]{article}

\usepackage[margin=1in]{geometry}
\usepackage[utf8]{inputenc}
\usepackage{exsheets}
\usepackage{centernot}
\usepackage{listings}

\DeclareInstance{exsheets-heading}{block-no-nr}{default}{
    attach = {
        main[l,vc]title[l,vc](0pt,0pt) ;
        main[r,vc]points[l,vc](\marginparsep,0pt)
    }
}
\RenewQuSolPair
{question}[headings=runin]
{solution}[headings=block-no-nr]

\SetupExSheets{
    counter-format=qu.,
    solution/print=true ,
    question/name=Feladat,
    solution/name=Megoldás.
}

\usepackage{tasks}
\usepackage[hungarian]{babel}
\usepackage{amsmath}
\usepackage{mathtools}
\usepackage{amsthm}
\usepackage[shortlabels]{enumitem}
\usepackage{amsfonts}
\usepackage{amssymb}
\usepackage{graphicx}
\usepackage{wrapfig}
\graphicspath{{./images/}}
\usepackage{float}
\usepackage{multicol}
\usepackage{tikz}
\usepackage{booktabs}
\usepackage{lstmisc}
\usepackage{cancel}
\usetikzlibrary{positioning,shapes,fit,arrows}
\tikzset{every picture/.style={line width=0.75pt}} %set default line width to 0.75pt

\title{\huge{Analízis II} \\ \large 6. Házi feladat}
\author{Boda Bálint}
\date{2022. őszi félév}

\theoremstyle{definition}
\newtheorem{definition}{Definíció}
\newtheorem*{definition*}{Definíció}
\newtheorem*{remark}{Megjegyzés}
\newtheorem{theorem}{Tétel}
\newtheorem*{theorem*}{Tétel}
\newtheorem*{example}{Példa}

\DeclareMathOperator{\lf}{lf}
\DeclareMathOperator{\ps}{p(S)}
\DeclareMathOperator{\prim}{pr\acute \jmath m}

\SetupExSheets{solution/print=true}
\SetupExSheets{question/name=}
\SetupExSheets{headings=runin}

\begin{document}
    \maketitle
    \begin{question}
        Írja fel az
        \[
        f(x) \coloneq \sqrt{1+2x} \quad \left( x \in \left] -\frac{1}{2},+\infty\right[ \right) 
        \]
        függvény nulla pont körüli második Taylor-polinomját! Adjon becslést az $ |f(x) - T_{2,0}f(x)| $ hibára a $ \left[ -\frac{5}{18},\frac{1}{4} \right]  $ intervallumon.
    \end{question}
	\begin{solution}
		\begin{align*}
			f'(x) &= \frac{1}{2\sqrt{1+2x}} \cdot 2 = \frac{1}{\sqrt{1+2x}} = (1+2x)^{-\frac{1}{2}} \\
			f''(x) &= -\frac{1}{2} \cdot (1+2x)^{-\frac{3}{2}} \cdot 2 = - (1+2x)^{-\frac{3}{2}}
		\end{align*}
	Ekkor,
	\[
	T_{2,0}f(x) = f(0) + \frac{f'(0)}{1\factorial} \cdot x + \frac{f''(0)}{2\factorial} \cdot x^2 = 1 + x - \frac{1}{2} x^2
	\]
	\[
	f'''(x) = \frac{3}{2} \cdot (1+2x)^{-\frac{5}{2}} \cdot 2 = \frac{3}{\sqrt{(1+2x)^5}}
	\]
	\[
	|f(x) - T_{2,0}f(x)| = \frac{1}{3\factorial} \cdot \frac{3}{\sqrt{\left(1 - \frac{5}{9}\right)^5}} \cdot \left( \frac{1}{4} \right)^{3} = \frac{1}{ \cancel{6}^{~2} } \cdot \frac{ 3^{\cancel{6}^{~5}} }{2^5} \cdot \frac{1}{2^6} = \frac{3^5}{2^{12}}
	\]
	\end{solution}
	\begin{question}
		Adja meg a következő függvények a pont körüli Taylor-sorát!
		\begin{tasks}(2)
			\task $f(x) \coloneq 2^x \quad (x \in \mathbb{R}), \; a = 1 $
			\task $f(x) \coloneq \ln{(x^2+1)} \quad (x \in \mathbb{R}), \; a = 0$
		\end{tasks}
	\end{question}
	\begin{solution}
		\begin{tasks}
			\task
			\[
			f'(x) = 2^x \cdot \ln{2} \qquad f''(x) = 2^x \cdot \ln^{2}{2} \qquad f'''(x) = 2^x \cdot \ln^{3}{2}
			\]
			\begin{align*}
				T_{1}f(x) &\coloneq 2 + 2\ln{2} (x-1) + \left( \ln^{2}{2} \right)  (x-1)^2 + \frac{\cancel{2}\ \left(  \ln^{3}{2} \right) (x-1)^3 }{\cancel{6}^{~3}} + \cdots \\
				&= \sum_{k=0}^{+\infty}{\frac{2 \cdot \left( \ln{2} \right)^{k}}{k \factorial} \cdot (x-1)^k}
			\end{align*}
			\task $ f(x) \coloneq \ln{(x^2+1)} \quad (x \in \mathbb{R}), \; a = 0$
			\begin{align*}
				f'(x) &= \frac{1}{x^2+1} \cdot 2x = \frac{2x}{x^2+1} \\
				f''(x) &= \frac{2(x^2+1)-2x(2x)}{(x^2+1)^2} = \frac{2-2x^2}{(x^2+1)^2} = \frac{2(1-x^2)}{(x^2+1)^2} \\
				f'''(x) &= \frac{-4x(x^4+2x^2+1)-(2-2x^2)(4x^3+4x)}{(x^2+1)^4} \\
				&= \frac{\cancel{-4x^5}-\cancel{8x^3}-4x-8x^3-8x+\cancel{8}^{~4}x^5+\cancel{8x^3}}{(x^2+1)^4} = \frac{4x^5-8x^3-12x}{(x^2+1)^4} \\
				f^{(4)}(x) &= \frac{(20x^4-24x^2-12)(x^2+1)^4-(4x^5-8x^3-12x)(3(x^2+1)(2x))}{(x^2+1)^8} \\
			\end{align*}
			\begin{align*}
				T_0f(x) &= 0 + 0 + x^2 + 0 + \frac{-12}{24}x^4 + 0 + \cdots = x^2 - \frac{1}{2}x^4 + \frac{1}{3}x^6 - \frac{1}{4}x^8 + \cdots \\
				&= \sum_{k=1}^{+\infty}{\frac{(-1)^{k+1}}{k} \cdot x^{2k}}
			\end{align*}
		\end{tasks}
	\end{solution}
\end{document}