%! Author = Valentinusz
%! Date = 2022. 10. 29.

\documentclass[a4paper,12pt]{article}

\usepackage[margin=1in]{geometry}

\usepackage[utf8]{inputenc}
\usepackage{exsheets}
\usepackage{centernot}
\usepackage{listings}
\usepackage{stuki}

\DeclareInstance{exsheets-heading}{block-no-nr}{default}{
    attach = {
        main[l,vc]title[l,vc](0pt,0pt) ;
        main[r,vc]points[l,vc](\marginparsep,0pt)
    }
}

\RenewQuSolPair
{question}[headings=runin]
{solution}[headings=block-no-nr]

\SetupExSheets{
    counter-format=qu.,
    solution/print=true ,
    question/name=Feladat,
    solution/name=Megoldás.
}

\usepackage{tasks}
\usepackage[hungarian]{babel}
\usepackage{amsmath}
\usepackage{mathtools}
\usepackage{amsthm}
\usepackage[shortlabels]{enumitem}
\usepackage{amsfonts}
\usepackage{amssymb}
\usepackage{graphicx}
\usepackage{wrapfig}
\graphicspath{{./images/}}
\usepackage{float}
\usepackage{multicol}
\usepackage{tikz}
\usepackage{booktabs}
\usepackage{lstmisc}
\usetikzlibrary{positioning,shapes,fit,arrows}
\tikzset{every picture/.style={line width=0.75pt}} %set default line width to 0.75pt

\title{\huge{Programozáselmélet} \\ \large 7. gyakorlat}
\author{Boda Bálint}
\date{2022. őszi félév}

\theoremstyle{definition}
\newtheorem{definition}{Definíció}
\newtheorem*{definition*}{Definíció}
\newtheorem{notation}{Jelölés}
\newtheorem*{notation*}{Jelölés}
\newtheorem*{remark}{Megjegyzés}
\newtheorem{theorem}{Tétel}
\newtheorem*{theorem*}{Tétel}
\newtheorem*{example}{Példa}

\DeclareMathOperator{\lf}{lf}
\DeclareMathOperator{\ps}{p(S)}
\DeclareMathOperator{\prim}{prim}

\SetupExSheets{solution/print=true}
\SetupExSheets{question/name=}
\SetupExSheets{headings=runin}

\begin{document}
    \maketitle
    \begin{notation*}[$\tau(\alpha)$] Legyen $\alpha$ egy véges hosszú sorozat, ekkor $\tau(\alpha) = \alpha_{|\alpha|}$. \end{notation*}
    \begin{definition*}[$\otimes$] Legyen $\alpha \in \bar{A}$ és $\beta \in \left(\bar{A} \cup \left\{ fail \right\} \right)^{**} $, nemüres sorozatok úgy, hogy ${\tau(\alpha) = \beta_1}$. Ekkor $ \alpha \otimes \beta $ jelöli a sorozatok összefűzéséből $ \beta $ első elemének elhagyásával kapott sorozatot.
    \end{definition*}
   	\begin{notation*}[$\otimes_n$] $\otimes_n(\alpha, \beta, ...)$-el jelöljük $n \in \mathbb{N}^+ \cup \left\{\infty\right\} $ sorozat az előző definíció szerinti egymás utáni összefésülését.
    \end{notation*}
	\section{Szekvencia}
    \begin{definition*}[Szekvencia]
    	Legyen $A$ egy tetszőleges állapottér és legyenek $ S_1, S_2, \subseteq A \times \left(\bar{A} \cup \left\{ fail \right\} \right)^{**}$ tetszőleges programok. Ekkor $(S_1;S_2)$ szekvenciának nevezzük azt a programot ami $S_1$ és $S_2$ egymást követő végrehajtásából jön létre.
    	\begin{align*}
    		(S_1;S_2)(a) =& \left\lbrace \alpha \in \bar{A}^{\infty} \; | \; \alpha \in S_1(a) \right\rbrace \; \cup \\
    		& \left\lbrace \alpha \in \left(\bar{A} \cup \left\{ fail \right\} \right)^{*} \; | \; \alpha \in S_1(a) \; \land \; \tau(\alpha) = fail \right\rbrace \; \cup \\
    		& \left\lbrace \alpha \otimes \beta \in \left(\bar{A} \cup \left\{ fail \right\} \right)^{**} \; | \; \alpha \in S_1(a) \, \land \, |\alpha| < \infty \, \land \, \tau(\alpha) \ne fail \, \land \, \beta \in S_2(\tau(\alpha)) \right\rbrace
    	\end{align*}
    A szekvencia struktogramja:
    \begin{stuki*}[3cm]{$ S_1;S_2 $}
    	\stm*{$S_1$}
    	\stm*{$S_2$}
    \end{stuki*}
    \end{definition*}
	\begin{example}
		Legyen $A = (x:\mathbb{Z}),\; a = \left\lbrace x:5 \right\rbrace $ és legyenek $S_1 \coloneq (x \coloneq x + 1), S_2 \coloneq (x \coloneq x + 2) $ programok. Ekkor $ (S_1;S_2)(\left\lbrace x:5 \right\rbrace) = \left\lbrace \left\langle  \left\lbrace x:5 \right\rbrace , \left\lbrace x:6 \right\rbrace, \left\lbrace x:8 \right\rbrace \right\rangle \right\rbrace $.
	\end{example}

	\newpage
	\section{Elágazás}
	
	\begin{definition*}[Elágazás]
		Legyen $A$ egy közös alap-állapottere az $S_1,...,S_n$ programoknak. Legyenek $\pi_1,...\pi_n \in A \rightarrow \mathbb{L}$ logikai függvények. Ekkor az $IF \subseteq A \times \left(\bar{A} \cup \left\{ fail \right\} \right)^{**}$ programot az $S_i$ programokból képzett $\pi_i$ feltételek által meghatározott elágazásnak nevezzük és $ \left( \pi_1:S_1,...\pi_n:S_n \right) $-nel jelöljük, ha
		\[ \forall a \in A: IF(a) = \omega_0(a) \cup \left( \bigcup_{i=1}^{n}{\omega_i(a)} \right)  \]
		ahol $ \forall i \in [1..n]: $
		\[
		\omega_i(a) = \begin{cases}
			S_i(a),& a \in D_{\pi_i} \, \land \ \pi_i(a) \\
			\emptyset,& a \in D_{\pi_i} \, \land \, \lnot \pi_i(a) \\
			\left\lbrace \left\langle a, fail \right\rangle  \right\rbrace,&  a \notin D_{\pi_i}
		\end{cases}
		\]
		és
		\[
		\omega_0(a) = \begin{cases}
			\left\lbrace \left\langle a, fail \right\rangle  \right\rbrace,&  \forall i \in [1..n]: (a \in D_{\pi_i} \, \land \lnot \pi_i(a)) \\
			\emptyset,& \text{különben}
		\end{cases}
		\]
		\begin{remark}
			A definíció a következőket fejezi ki:
			\begin{itemize}
				\item $\omega_0(a)$: Ha az adott állapot egyik logikai függvényt sem elégíti ki a végrehajtás legyen hibás.
				\item $\omega_i(a)$: Ha az adott állapot esetén a $\pi_i$ logikai függvény  \begin{itemize}
					\item nem kiértékelhető: a végrehajtás legyen hibás.
					\item nem teljesül: a végrehajtás érjen véget $S_i$ végrehajtása nélkül.
					\item teljesül: hajtsuk végre az $S_i$ programot.
				\end{itemize}
				\item $\bigcup_{i=1}^{n}{\omega_i(a)}$: az összes logikai függvényt vizsgáljuk meg
			\end{itemize}
		\end{remark}
	Az elágazás struktogramja:
	\begin{stuki*}[5cm]{$IF$}
		\begin{CASE}{3}{3}
			\WHEN{\stm{\pi_1}}
			\stm[1]{S_1}
			\WHEN*{\stm{...}}
			\stm[1]{...}
			\WHEN{\stm{\pi_n}}
			\stm[1]{S_n}
		\end{CASE}
	\end{stuki*}
	\end{definition*}
	\newpage
	\begin{question} Legyen $A = [1..6]$ és legyenek $ S_1, S_2 \subseteq A \times \left(\bar{A} \cup \left\{ fail \right\} \right)^{**} $ a következő programok:
		\begin{align*}
			S_1 &=\left\{
			\begin{array}{lll}
				1 \rightarrow \left< 1,4,3 \right>, & 1 \rightarrow \left< 1,2,4 \right>,& 2 \rightarrow \left< 2,2,\dots \right>, \\
				2 \rightarrow \left< 2,1,4,6 \right>,     & 3 \rightarrow \left< 3,5,1 \right>, & 4 \rightarrow \left< 4,5,3 \right>, \\
				5 \rightarrow \left< 5,1,fail \right>,     & 6 \rightarrow \left< 6,3,1,5 \right> &
			\end{array}
			\right\}
			\\[12pt]
			S_2 &=\left\{
			\begin{array}{lll}
				1 \rightarrow \left< 1,3,2 \right>, & 1 \rightarrow \left< 1,2,4 \right>,& 2 \rightarrow \left< 2,6 \right>, \\
				3 \rightarrow \left< 3,4 \right>,     & 4 \rightarrow \left< 4,fail \right>, & 4 \rightarrow \left< 4,5,1 \right>, \\
				5 \rightarrow \left< 5 \right>,     & 6 \rightarrow \left< 6,4,3,2 \right> &
			\end{array}
			\right\}
		\end{align*}
	\begin{tasks}
		\task Határozza meg az $(S_1;S_2)$ szekvenciát!
		\task Legyenek $\pi_1, \pi_2 \in A \rightarrow \mathbb{L}$ logikai függvények, úgy hogy:
		\begin{align*}
			\pi_1&=\left\lbrace (1,igaz),(2,igaz),(4,igaz),(5,hamis),(6,hamis) \right\rbrace \\
			\pi_2&=\left\lbrace (1,igaz),(2,hamis),(3,igaz),(4,igaz),(5,hamis) \right\rbrace.
		\end{align*}
		Határozza meg a $(\pi_1:S_1, \pi_2:S_2)$ elágazást!
	\end{tasks}
	\end{question}
	\begin{solution}
		\begin{tasks}
			\task{\[ (S_1;S_2) = \left\lbrace 
				\begin{array}{lll}
				1 \rightarrow \left< 1,4,\mathbf{3},4 \right>, & 1 \rightarrow \left< 1,2,\mathbf{4},fail \right>, & 1 \rightarrow \left< 1,2,\mathbf{4},5,1 \right>, \\
				2 \rightarrow \left< 2,2,\dots \right>, & 2 \rightarrow \left< 2,1,4,\mathbf{6},4,3,2 \right>, & \\
				3 \rightarrow \left< 3,5,\mathbf{1},3,2 \right>, & 3 \rightarrow \left< 3,5,\mathbf{1},2,4 \right>, & \\
				4 \rightarrow \left< 4,5,\mathbf{3},4 \right>, & & \\ 
				5 \rightarrow \left< 5,1,fail \right>, & & \\
				6 \rightarrow \left< 6,3,1,\mathbf{5} \right> 
				\end{array} \right\rbrace 
			\]}
			\task{\[ (\pi_1\!:\!S_1, \pi_2\!:\! S_2) = \left\lbrace 
				\begin{array}{lll}
					1 \rightarrow \left< 1,4,3 \right>, & 1 \rightarrow \left< 1,2,4 \right>,& 1 \rightarrow \left< 1,3,2 \right>, \\
					2 \rightarrow \left< 2,2,\dots \right>, & 2 \rightarrow \left< 2,1,4,6 \right>,     & \\
					3 \rightarrow \left< 3,\mathbf{fail} \right>, & 3 \rightarrow \left< 3,4 \right>, & \\
					4 \rightarrow \left< 4,5,3 \right>, & 4 \rightarrow \left< 4,fail \right>, & 4 \rightarrow \left< 4,5,1 \right>, \\
					5 \rightarrow \left< 5,\underline{fail} \right>, & & \\
					6 \rightarrow \left< 6, \underline{\mathbf{fail}} \right> & & 
				\end{array} \right\rbrace 
			\]}
		 	\noindent
			A félkövérrel kiemelt $fail$ állapotok azért következnek be mert az adott $\pi_i$ logikai függvény nincs értelmezve az adott állapotban, az aláhúzottak pedig azért, mert semelyik logikai függvény nem teljesül az adott állapotban.
		\end{tasks}
	\end{solution}
	\newpage
	\section{Ciklus}
		\begin{definition*}[Ciklus]
		Legyen $\pi \in A \rightarrow \mathbb{L}$ feltétel és $S_0 \subseteq A \times \left(\bar{A} \cup \left\{ fail \right\} \right)^{**} $ program. A $DO \subseteq A \times \left(\bar{A} \cup \left\{ fail \right\} \right)^{**} $ programot az $S_0$ programból $\pi$ feltétellel képzett ciklusnak nevezzük és $(\pi, S_0)$-val jelöljük, ha
		\[ \forall a \in A: DO(a) = \begin{cases}
			(S_0;DO)(a), & a \in D_{\pi} \, \land \, \pi(a) \\
			\left\lbrace \left\langle a \right\rangle  \right\rbrace , & a \in D_{\pi} \, \land \, \lnot \pi(a) \\
			\left\lbrace \left\langle a, fail \right\rangle  \right\rbrace,&  a \notin D_{\pi}
		\end{cases}\]
		A ciklus struktogramja:
		\begin{stuki*}[3cm]{DO}
			\begin{WHILE}{1}{\stm{\pi}}
				\stm{S_0}
			\end{WHILE}
		\end{stuki*}
	\end{definition*}
	\setcounter{question}{3}
	\begin{question}
		Legyen $A = [1..5], S_0 \subseteq A \times \left(\bar{A} \cup \left\{ fail \right\} \right)^{**}$ program, továbbá $ \pi: A \rightarrow \mathbb{L} $ feltétel úgy, hogy $ \lceil \pi \rceil \left\lbrace 1,2,3,4 \right\rbrace $ és
		\[ S_0 = \left\{ \begin{array}{lll}
			1 \rightarrow \left< 1,2,4 \right>, & 2 \rightarrow \left< 2 \right>,& 3 \rightarrow \left< 3,4,2 \right>, \\
			3 \rightarrow \left< 3,5 \right>,     & 3 \rightarrow \left< 3,3,3,\dots \right>, & 4 \rightarrow \left< 4,5,3,4 \right>, \\
			4 \rightarrow \left< 4,1,3 \right>,     & 5 \rightarrow \left< 5,5,\dots \right> &
		\end{array} \right\} \]
		Határozza meg a $(\pi, S_0)$ ciklust!
	\end{question}
	\begin{solution}
		A ciklus egy $ \alpha $ sorozata csak akkor lesz véges ha $ \tau(\alpha)=5 $.
		\[ (\pi, S_0) = \left\{ \begin{array}{ll}
			1 \rightarrow \left< 1,2,\mathbf{4},(5,3,\mathbf{4}) \cdot \infty \right>, &
			1 \rightarrow \left< 1,2, \mathbf{4}, (5, 3, \mathbf{4}) \cdot a, 1, \mathbf{3}, 3, \dots \right>, \\
			1 \rightarrow \left< 1,2, \mathbf{4}, (5, 3, \mathbf{4}) \cdot b, 1, \mathbf{3}, 4, \textbf{2}, 2, \dots \right>, &
			1 \rightarrow \left< 1,2, \textbf{4}, (5, 3, \mathbf{4}) \cdot c, 1, \mathbf{3}, 5 \right>, \\
			1 \rightarrow \left< 1,2, \mathbf{4}, 1, \mathbf{3}, 4, \mathbf{2}, 2, \dots\right>, &
			1 \rightarrow \left< 1,2, \mathbf{4}, 1, \mathbf{3}, 3, \dots \right> \\
			1 \rightarrow \left< 1,2, \mathbf{4}, 1, \mathbf{3}, 5 \right>, &
			2 \rightarrow \left< 2,2,\dots \right>, \\
			3 \rightarrow \left< 3,4,\mathbf{2},2,\dots \right>, &
			3 \rightarrow \left< 3,5 \right>, \\
			3 \rightarrow \left< 3,3,\dots \right>, &
			4 \rightarrow \left< 4,5,3,\mathbf{4}\cdot \infty \right> \\
			4 \rightarrow \left< 4,(5,3,\mathbf{4}) \cdot d, 1,\mathbf{3},4,\mathbf{2},2\dots \right> &
			4 \rightarrow \left< 4,(5,3,\mathbf{4}) \cdot e, 1,\mathbf{3},5 \right> \\
			4 \rightarrow \left< 4,(5,3,\mathbf{4}) \cdot f, 1,\mathbf{3},3,\dots \right> &
			5 \rightarrow \left< 5 \right>
		\end{array} \right\} \]
		\[ a,b,c,d,e,f \in \mathbb{N} \]   
	\end{solution}
	\setcounter{question}{5}
	\newpage
	\begin{question}
		Rajzolja fel a következő program struktogramját és határozza meg mit rendel az $ \left\lbrace i:2, n:12 \right\rbrace $ és $ \left\lbrace i:1, n:13 \right\rbrace $ állapotokhoz!
		\begin{align*}
			A &= (i: \mathbb{N}, n: \mathbb{N}) \\
			S &= (i \coloneq 1;\; DO(i \ne n, IF(2 | i \, \land \; i \le 12: i \coloneq i+1, \quad 2 \centernot | i \, \lor \, (2 | i \, \land \, 12 \le i < 20): i \coloneq i + 3)))
		\end{align*}
	\end{question}
	\begin{solution}
		\begin{stuki}[12cm]
			\stm{i \coloneq 1}
			\begin{WHILE}{2}{\stm{i \ne n}}
				\begin{CASE}{1}{2}
					\WHEN{\stm{2 | i \, \land \, i \le 12}}
					\stm[0.5]{i \coloneq i + 1}
					\WHEN{\stm{2 \centernot | i \, \lor \, (2 | i \, \land 12 \le i < 20)}}
					\stm[0.5]{i \coloneq i + 3}
				\end{CASE}
			\end{WHILE}
		\end{stuki}
	\[
	(2,12) \rightarrow \left\langle (2,12), (1,12), (4,12), (5,12), (8,12), (9,12), (12,12) \right\rangle
	\]
	\[
	(2,13) \rightarrow \left\langle (1,13), (1,13), (1,13), (4,13), (5,13), (8,13), (9,13), \underline{(12, 13)},\text{elágazás} \right\rangle
	\]
	Mivel mindkét logikai függvény teljesül a program az aláhúzással jelölt állapot esetén a program kétféleképpen folytatódhat.
	Bal oldali ág:
	\[
	(2,13) \rightarrow \left\langle \dots, \underline{(12, 13)}, (13,13)\right\rangle
	\]
	Jobb oldali ág:
	\[
	(2,13) \rightarrow \left\langle \dots, \underline{(12, 13)}, (15,13), (18, 13), (21, 13), (24,13), fail \right\rangle
	\]
	\end{solution}
	\setcounter{question}{-1}
	\begin{question} Készítsen programot az alapvető programkonstrukciók felhasználásával, mely ekvivalens
		\begin{tasks}(2)
			\task a $SKIP$-el!
			\task az $ABORT$-al!
		\end{tasks}
	\end{question}
	\begin{solution}
		\begin{tasks}(2)
			\task{
				\begin{stuki}[3cm]
				\begin{WHILE}{2}{\stm{HAMIS}}
					\stm{S}
				\end{WHILE}
			\end{stuki}
			}
			\task{
			\begin{stuki}[3cm]
				\begin{CASE}{1}{1}
					\WHEN{\stm{HAMIS}}
					\stm[1]{S}
				\end{CASE}
			\end{stuki}
			}
		\end{tasks}
	(Ahol $S$ tetszőleges program.)
	\end{solution}
\end{document}