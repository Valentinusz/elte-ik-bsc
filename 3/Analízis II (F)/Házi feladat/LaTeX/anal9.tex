\documentclass[a4paper,12pt]{article}

\usepackage[margin=1in]{geometry}
\usepackage[utf8]{inputenc}
\usepackage{exsheets}
\usepackage{centernot}
\usepackage{listings}

\DeclareInstance{exsheets-heading}{block-no-nr}{default}{
    attach = {
        main[l,vc]title[l,vc](0pt,0pt) ;
        main[r,vc]points[l,vc](\marginparsep,0pt)
    }
}
\RenewQuSolPair
{question}[headings=runin]
{solution}[headings=block-no-nr]

\SetupExSheets{
    counter-format=qu.,
    solution/print=true ,
    question/name=Feladat,
    solution/name=Megoldás.
}

\usepackage{tasks}
\usepackage[hungarian]{babel}
\usepackage{amsmath}
\usepackage{mathtools}
\usepackage{amsthm}
\usepackage[shortlabels]{enumitem}
\usepackage{amsfonts}
\usepackage{amssymb}
\usepackage{graphicx}
\usepackage{wrapfig}
\graphicspath{{./images/}}
\usepackage{float}
\usepackage{multicol}
\usepackage{tikz}
\usepackage{booktabs}
\usepackage{lstmisc}
\usepackage{cancel}
\usepackage{mathtools}
\usetikzlibrary{positioning,shapes,fit,arrows}
\tikzset{every picture/.style={line width=0.75pt}} %set default line width to 0.75pt

\title{\huge{Analízis II} \\ \large 9. Házi feladat}
\author{Boda Bálint}
\date{2022. őszi félév}

\theoremstyle{definition}
\newtheorem{definition}{Definíció}
\newtheorem*{definition*}{Definíció}
\newtheorem*{remark}{Megjegyzés}
\newtheorem{theorem}{Tétel}
\newtheorem*{theorem*}{Tétel}
\newtheorem*{example}{Példa}

\DeclareMathOperator{\lf}{lf}
\DeclareMathOperator{\ps}{p(S)}
\DeclareMathOperator{\prim}{pr\acute \jmath m}
\DeclareMathOperator{\tg}{tg}
\DeclareMathOperator{\arctg}{arctg}

\SetupExSheets{solution/print=true}
\SetupExSheets{question/name=}
\SetupExSheets{headings=runin}

\usepackage{pgfplots}
\pgfplotsset{compat=1.15}
\usepackage{mathrsfs}

\begin{document}
	\maketitle
	\begin{question}
		Számítsa ki a következő határozatlan integrálokat!
		\begin{tasks}
			\task $ \displaystyle \int{\frac{\sqrt{3x-1}}{x} \mathop{dx}} $ \\[8pt]
			A függvény értelmezve van, ha $ x \ge \frac{1}{3} $. Használjuk a $ t = \sqrt{3x-1} $ helyettesítést. Ekkor
			\begin{align*}
				&x = \frac{t^2+1}{3} \eqcolon g(t) \; \left( t \in \left[ 0,+\infty \right)  \right) \\
				&g'(t) = \frac{2}{3}t \ge 0 \; \left( t \in \left[ 0,+\infty \right)  \right) \implies g \text{ szigorúan monoton nő} \implies g\text{ invertálható} \\
				&g^{-1}(x) = \sqrt{3x-1} = t \; \left( x \in \left[ \frac{1}{3}, +\infty \right)  \right)
			\end{align*}
			A második helyettesítési szabály alapján, ha $ x \in \left[ \frac{1}{3}, +\infty \right) $:
			\begin{align*}
				\int{\frac{\sqrt{3x-1}}{x} \mathop{dx}} &= \int{\frac{3}{t^2+1} \cdot t \cdot \frac{2}{3}t \mathop{dt}} = \int{\frac{2t^2}{t^2+1} \mathop{dt}} = 2 \int{\frac{t^2+1-1}{t^2+1} \mathop{dt}} \\
				&= 2 \int{1 \mathop{dt}} - 2\int{\frac{1}{t^2+1} \mathop{dt}} = 2t - 2\arctg{t} \\
				&= 2 \sqrt{3x-1} - 2\arctg{\sqrt{3x-1}} + c
			\end{align*}
			\task $ \displaystyle \int \limits_{0}^{+\infty}{e^{-4x} \mathop{dx}} \quad F = \displaystyle \int{e^{-4x} \mathop{dx}} = \frac{e^{-4x}}{-4} $
			\begin{align*}
				\int \limits_{0}^{+\infty}{e^{-4x} \mathop{dx}} &= \lim\limits_{t \rightarrow + \infty} \left( \int \limits_{0}^{t}{e^{-4x} \mathop{dx}} \right) = \lim\limits_{t \rightarrow + \infty} \left( F(t) - F(0) \right) \\
				&= \lim\limits_{t \rightarrow + \infty} \left( \frac{e^{-4x}}{-4} - \left( - \frac{1}{4} \right) \right) = 0 + \frac{1}{4} = \frac{1}{4}
			\end{align*}
			\task $ \displaystyle \int\limits_  {\ln 4}^{\ln 8}{\frac{e^x}{e^{2x}-4} \mathop{dx}}  $ \\[8pt]
			A függvény a teljes intervallumon értelmezve van ezért gond nélkül meghatározhatjuk a határozott integrált.
			\\[4pt]
			Ehhez alkalmazzuk a $t = e^x \; \left( x \in \left[ \ln 4, \ln 8 \right]  \right) $ helyettesítést.  Ekkor
			\begin{align*}
				&x = \ln(t) \eqcolon g(t) \; \left( t \in \left[ 4, 8 \right]  \right) \\
				&g'(t) = \frac{1}{t} > 0 \; \left( t \in \left[4,8\right]\right) \implies g \text{ szigorúan monoton nő} \implies g\text{ invertálható} \\
				&g^{-1}(x) = e^x = t \; \left( x \in \left[ \ln 4, \ln 8 \right]  \right)
			\end{align*}
			A második helyettesítési szabály alapján, ha $ x \in \left[ \ln 4, \ln 8 \right] $:
			\begin{align*}
				\int{\frac{e^x}{e^{2x}-4} \mathop{dx}} &= \int{\frac{t}{t^2-4} \cdot \frac{1}{t} \mathop{dt}} = \int{\frac{1}{(t+2)(t-2)} \mathop{dt}} = \int{\frac{A}{(t+2)} + \frac{B}{(t-2)} \mathop{dt}}
			\end{align*}
			A parciális törtekre bontás módszerével:
			\begin{align*}
				A(t-2)+B(t+2) &= 0x+1 \\
				At-2A+Bt+2B &= 0x+1 \\
			\end{align*}
			\\[-40pt]
			Melyből a következő egyenletrendszer adódik:
			\begin{align*}
				-2A + 2B &= 1 \iff 4B = 1 \iff B = \frac{1}{4} \\
				A+B &= 0 \iff A = -B \iff A = -\frac{1}{4}
			\end{align*}
			\\[-40pt]
			\begin{align*}
				\int{\frac{1}{(t+2)(t-2)} \mathop{dt}} &= -\frac{1}{4} \int{\frac{1}{(t+2)} \mathop{dt} + \frac{1}{4} \int{\frac{1}{(t-2)}} \mathop{dt}} \\
				&= \frac{1}{4} \ln(t-2) - \frac{1}{4} \ln (t+2) \eqcolon G
			\end{align*}
			A Newton-Leibniz formula alapján:
			\begin{align*}
				\int\limits_{\ln 4}^{\ln 8}{\frac{e^x}{e^{2x}-4} \mathop{dx}} &= \int\limits_{4}^{8}{\frac{1}{(t+2)(t-2)} \mathop{dt}} = G(8) - G(4) \\
				&= \frac{1}{4} \ln 6 - \frac{1}{4} \ln 10 - \left(\frac{1}{4} \ln 2 - \frac{1}{4} \ln 6\right) \\
				&= \frac{1}{2} \ln 6 - \frac{1}{4} \ln 10 - \frac{1}{4} \ln 2 = \frac{1}{4} \left( 2\ln 6 - \ln 10 - \ln 2 \right) \\
				&= \frac{1}{4} \left( \ln 36 - \ln 10 - \ln 2 \right) = \frac{1}{4} ( \ln{3,6} - \ln 2 ) = \frac{1}{4} \ln 1,8 = \ln \sqrt[4]{1,8}
			\end{align*}
			\begin{remark}
				A feladat természetesen megoldható úgy is, hogy visszahelyettesítéssel kiszámoljuk az eredeti függvény primitív függvényét és arra alkalmazzuk a Newton-Leibniz formulát.
			\end{remark}
		\end{tasks} 
	\end{question}
	\newpage
	\begin{question}
		Számítsa ki az $y = x^2$, $y = \frac{x^2}{2}$, és az $y = 2x$ egyenletű görbék által közrezárt korlátos síkidom területét.  \\[8pt]
		Jelölje $f$ az $x^2$, $g$ az $\frac{x^2}{2}$ és $h$ a $2x$ görbét.
		A metszéspontok meghatározásához a következő egyenleteket kell megoldani:
		\begin{align*}
			x^2 &= \frac{x^2}{2} \iff 0,5x^2=0    & x=0 \\
			x^2 &= 2x \iff x^2-2x=0 \iff x(x-2)=0 & x_1=2, x_2=0 \\
			\frac{x^2}{2} &= 2x \iff x(x-4)=0     & x_1=4, x_2=0
		\end{align*}
		Így a metszéspontok: $M_1(0,0)=M_3=M_5,\; M_2(2,4),\; M_3(4,8)$. 
			\definecolor{ududff}{rgb}{0.30196078431372547,0.30196078431372547,1.}
		\definecolor{ffwwqq}{rgb}{1.,0.4,0.}
		\definecolor{qqqqff}{rgb}{0.,0.,1.}
		\definecolor{ccqqqq}{rgb}{0.8,0.,0.}
		\definecolor{qqwuqq}{rgb}{0.,0.39215686274509803,0.}
		\begin{wrapfigure}[14]{h}{6cm}
			\begin{tikzpicture}[line cap=round,line join=round,>=triangle 45,x=1.0cm,y=1.0cm]
				\begin{axis}[
					x=1.0cm,y=1.0cm,
					axis lines=middle,
					ymajorgrids=true,
					xmajorgrids=true,
					xmin=-1.0,
					xmax=5.0,
					ymin=-1.0,
					ymax=9.0,
					xtick={-0.0,2.0,...,4.0},
					ytick={-0.0,2.0,...,8.0},]
					\clip(-1.,-1.) rectangle (5.,9.);
					\draw[line width=0.8pt,color=ffwwqq,fill=ffwwqq,fill opacity=0.10000000149011612] {[smooth,samples=50,domain=0.0:2.0] plot(\x,{(\x)^(2)})} -- (2.,2.) {[smooth,samples=50,domain=2.0:0.0] -- plot(\x,{(\x)^(2)/2})} -- (0.,0.) -- cycle;
					\draw[line width=0.8pt,color=ffwwqq,fill=ffwwqq,fill opacity=0.10000000149011612] {[smooth,samples=50,domain=2.0:4.0] plot(\x,{2*(\x)})} -- (4.,8.) {[smooth,samples=50,domain=4.0:2.0] -- plot(\x,{(\x)^(2)/2})} -- (2.,4.) -- cycle;
					\draw[line width=2.pt,color=qqwuqq,smooth,samples=100,domain=-1.0:5.0] plot(\x,{(\x)^(2)});
					\draw[line width=2.pt,color=ccqqqq,smooth,samples=100,domain=-1.0:5.0] plot(\x,{(\x)^(2)/2});
					\draw[line width=2.pt,color=qqqqff,smooth,samples=100,domain=-1.0:5.0] plot(\x,{2*(\x)});
					\draw [line width=1.2pt,dash pattern=on 3pt off 3pt] (0.,4.)-- (2.,4.);
					\draw [line width=1.2pt,dash pattern=on 3pt off 3pt] (2.,4.)-- (2.,0.);
					\draw [line width=1.2pt,dash pattern=on 3pt off 3pt] (0.,8.)-- (4.,8.);
					\draw [line width=1.2pt,dash pattern=on 3pt off 3pt] (4.,0.)-- (4.,8.);
					\begin{scriptsize}
						\draw[color=qqwuqq] (-0.43436984684128715,0.5615175573244229) node {$f$};
						\draw[color=ccqqqq] (-0.8428080631942543,0.11478845679330517) node {$g$};
						\draw[color=qqqqff] (-0.3833150697971663,-0.3064132665646058) node {$h$};
						\draw[color=ffwwqq] (1.6014393877930333,1.7804498173450443) node {$T_1$};
						\draw[color=ffwwqq] (2.5842438458923604,4.486351797704957) node {$T_2$};
						\draw [fill=ududff] (0.,0.) circle (2.5pt);
						\draw[color=ududff] (0.26125149038485995,-0.3766135537909243) node {$M_1$};
						\draw [fill=ududff] (2.,4.) circle (2.5pt);
						\draw[color=ududff] (1.741840024664366,4.294896468905907) node {$M_2$};
						\draw [fill=ududff] (4.,8.) circle (2.5pt);
						\draw[color=ududff] (4.2945788768704105,7.753856075875419) node {$M_3$};
					\end{scriptsize}
				\end{axis}
			\end{tikzpicture}

		\end{wrapfigure}
		A keresett síkidomot alulról a $g$ görbe, felülről a $[0,2]$ intervallumon az $f$ görbe a $[2,4]$ intervallumon pedig a $h$ görbe határolja. Így a következő két terület adódik:
		\begin{align*}
			T_1 &= \left\lbrace (x,y) \in \mathbb{R}^2 \; | \; 0 \le x \le 2, \, \frac{x^2}{2} \le y \le x^2 \right\rbrace \\
			T_2 &= \left\lbrace (x,y) \in \mathbb{R}^2 \; | \; 2 \le x \le 4, \, \frac{x^2}{2} \le y \le 2x \right\rbrace
		\end{align*}
		Melyeket integrálással könnyen meg tudunk határozni:
		\begin{align*}
			\int x^2 \mathop{dx} &= \frac{x^3}{3} + c \\
			\int \frac{x^2}{2} \mathop{dx} &= \frac{x^3}{6} + c \\
			\int 2x \mathop{dx} &= x^2 + c
		\end{align*}
		\begin{align*}
			T_1 &= \int\limits_{0}^{2}{x^2} \mathop{dx} - \int\limits_{0}^{2}{\frac{x^2}{2}} \mathop{dx} = \frac{8}{3} - 0 - \left( \frac{8}{6} -0 \right) = \frac{8}{6} \\
			T_2 &= \int\limits_{2}^{4}{2x \mathop{dx}} - \int\limits_{2}^{4}{\frac{x^2}{2} \mathop{dx}} = 16 - 4 - \left( \frac{64}{6} - \frac{8}{6} \right) = \frac{72}{6} - \frac{56}{6} = \frac{16}{6} \\
			T &= \frac{8}{6} + \frac{16}{6} = 4
		\end{align*}
	\end{question}
	\newpage
	\begin{question}
		Határozza meg az $f(x) \coloneq \sqrt{\arctg{x}} \quad \left(x \in \left[0,1\right]\right) $ függvény grafikonjának az x-tengely körüli megforgatásával adódó forgástest térfogatát!
		\begin{align*}
			V &= \pi \cdot \int\limits_{0}^{1}{\arctg x \mathop{dx}} \\
		\end{align*}
		\\[-45pt]
		Parciális integrálással:
		\begin{align*}
			\int{\arctg x \cdot 1 \mathop{dx}} &= \arctg x \cdot \int{1 \mathop{dx}} - \int{\frac{1}{1+x^2} \cdot x \mathop{dx}} = x \cdot \arctg x - \frac{1}{2} \int{\frac{2x}{1+x^2} \mathop{dx}} \\
			&= x \cdot \arctg x - \frac{1}{2} \cdot \ln(x^2+1) + c
		\end{align*}
		Így
		\begin{align*}
			V &= \pi \cdot \left( \arctg 1 - \frac{1}{2} \ln 2 - \left(0 \arctg 0 - \frac{1}{2} \ln 1\right) \right) = \pi \cdot \left( \frac{\pi}{4} - \frac{1}{2} \ln2 \right) = \frac{\pi^2}{4} - \frac{\pi}{2} \ln 2
		\end{align*}
	\end{question}
	\begin{question}
		Számítsa ki az $f(x) \coloneq x^{\frac{3}{2}} \quad \left(x \in \left[0,4\right]\right) $ függvény grafikonjának ívhosszát!
		\begin{align*}
			\ell &= \int\limits_{0}^{4}{\sqrt{1+\left[ f'(x) \right]^2} \mathop{dx}} \\
			\int{\sqrt{1+  \left( \frac{3}{2} x^{\frac{1}{2}} \right)^2 } \mathop{dx} } &= \int{\sqrt{1 + \frac{9}{4} x} \mathop{dx} } = \frac{3}{2} \int{\sqrt{1x+\frac{4}{9}} \mathop{dx} }
		\end{align*}
		Lineáris helyettesítéssel:
		\begin{align*}
			\frac{3}{2} \int{\left( {1x+\frac{4}{9}}\right)^\frac{1}{2} \mathop{dx} } = \frac{3}{2} \cdot \frac{\frac{2 \cdot \left(x + \frac{4}{9}\right)^\frac{3}{2}}{3}}{1} = \left(x + \frac{4}{9}\right)^\frac{3}{2}
		\end{align*}
		A Newton-Leibniz formula alapján:
		\begin{align*}
			\ell &= \int\limits_{0}^{4}{\left(x + \frac{4}{9}\right)^\frac{3}{2} \mathop{dx}} = \left(4 + \frac{4}{9}\right)^\frac{3}{2} - \left(\frac{4}{9}\right)^\frac{3}{2} = \frac{40^\frac{3}{2}}{27} - \frac{8}{27} = \frac{8 \cdot \sqrt{1000}}{27} - \frac{8}{27} \\
			&= \frac{8 \cdot \sqrt{1000}-8}{27}
		\end{align*}
	\end{question}
\end{document}