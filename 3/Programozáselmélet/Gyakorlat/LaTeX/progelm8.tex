\documentclass[a4paper,12pt]{article}

\usepackage[margin=1in]{geometry}
\usepackage[utf8]{inputenc}
\usepackage{exsheets}
\usepackage{centernot}
\usepackage{listings}

\DeclareInstance{exsheets-heading}{block-no-nr}{default}{
    attach = {
        main[l,vc]title[l,vc](0pt,0pt) ;
        main[r,vc]points[l,vc](\marginparsep,0pt)
    }
}
\RenewQuSolPair
{question}[headings=runin]
{solution}[headings=block-no-nr]

\SetupExSheets{
    counter-format=qu.,
    solution/print=true ,
    question/name=Feladat,
    solution/name=Megoldás.
}

\usepackage{tasks}
\usepackage[hungarian]{babel}
\usepackage{amsmath}
\usepackage{mathtools}
\usepackage{amsthm}
\usepackage[shortlabels]{enumitem}
\usepackage{amsfonts}
\usepackage{amssymb}
\usepackage{graphicx}
\usepackage{wrapfig}
\graphicspath{{./images/}}
\usepackage{float}
\usepackage{multicol}
\usepackage{tikz}
\usepackage{booktabs}
\usepackage{lstmisc}
\usepackage{stuki}
\usetikzlibrary{positioning,shapes,fit,arrows}
\tikzset{every picture/.style={line width=0.75pt}} %set default line width to 0.75pt

\title{\huge{Programozáselmélet} \\ \large 8. gyakorlat}
\author{Boda Bálint}
\date{2022. őszi félév}

\theoremstyle{definition}
\newtheorem{definition}{Definíció}
\newtheorem*{definition*}{Definíció}
\newtheorem*{remark}{Megjegyzés}
\newtheorem{theorem}{Tétel}
\newtheorem*{theorem*}{Tétel}
\newtheorem*{example}{Példa}

\DeclareMathOperator{\lf}{lf}
\DeclareMathOperator{\ps}{p(S)}
\DeclareMathOperator{\prim}{pr\acute \jmath m}

\SetupExSheets{solution/print=true}
\SetupExSheets{question/name=}
\SetupExSheets{headings=runin}

\begin{document}
    \maketitle
    \noindent
    Bár sok program helyessége könnyen belátható a megoldás definíciójával vagy a specifikáció tételével, könnyű meggondolni, hogy kellően bő állapotterekre ez a két módszer lehetetlenül sok számolást igényelne. Az alábbi tételek a specifikáció tételének $\forall b \in B: Q_b \implies \lf(S,R)) $ feltételének belátására használhatók.
    \section{A szekvencia levezetési szabálya}
    \begin{theorem*}
        Legyen $S=(S_1;S_2)$, ahol $A$ az $S_1$ és $S_2$ programok közös állapottere. Legyenek továbbá $Q, R, Q' $ logikai függvények $A$-n. Ekkor, ha
        \begin{enumerate}
        	\item $ Q \implies \lf(S_1,Q') $
        	\item $ Q' \implies \lf(S_2,R) $
        \end{enumerate}
    	akkor $ Q \implies \lf(S,R)) $.
    \end{theorem*}
	\begin{remark}
		A tétel azt fejezi ki, hogy a program egy olyan pontról ahol $Q$ igaz el tud jutni egy olyan pontra, ahol igaz $R$ egy közbülső $Q'$ logikai feltételen keresztül. 
	\end{remark}

	\section{Az elágazás levezetési szabálya}
	Legyen $IF = (\pi_1:S_1,\dots,\pi_n:S_n)$ a közös $A$ állapotterű $S_i$ programokból képzett $A$ feletti $\pi_i$ logikai függvényekkel meghatározott elágazás. Legyenek továbbá $Q$ és $R$ logikai függvények. Ha
	\begin{enumerate}
		\item $ Q \implies \displaystyle \bigvee \limits _{i=1}^{n}{\pi_i}  $ \qquad (ha Q igaz legalább az egyik feltétel teljesül)
		\item $ Q \implies \displaystyle \bigwedge \limits _{i=1}^{n}{(\pi_i \lor \lnot \pi_i)} $ \qquad (ha Q igaz minden feltétel kiértékelhető)
		\item $ \forall i \in \left[ 1..n \right]: (Q \land \pi_i) \implies \lf(S_i,R) $
	\end{enumerate}
	akkor $ Q \implies \lf(IF,R) $.
	\newpage
	\section{A ciklus levezetési szabálya}
	\begin{definition*}[ciklusinvariáns]
		Legyen $DO = (\pi, S_0) $ egy ciklus az $A$ állapottér felett. Ekkor ciklusinvariánsnak nevezzük $P$ a  logikai feltétel, ha a $DO$ ciklus minden végrehajtása esetén $P$ igaz.
	\end{definition*}
	\begin{definition*}[termináló függvény]
		Legyen $DO = (\pi, S_0) $ egy ciklus az $A$ állapottér felett. Ekkor termináló függvénynek nevezzük a  $t: A \rightarrow \mathbb{Z}$ függvényt, ha a $DO$ ciklus minden végrehajtása esetén a $t$ függvénye értéke kisebb lesz mint az előző végrehajtás esetén.
	\end{definition*}
	\begin{theorem*}[ciklus levezetési szabálya]
		Legyen $DO = (\pi, S_0) $ egy ciklus az $A$ állapottér felett. Továbbá legyenek $P$, $Q$ és $R$ logikai függvények $A$-n és $t:A \rightarrow \mathbb{Z}$ függvény adottak. Ha
		\begin{enumerate}
			\item $ Q \implies P $ \\ (ha az előfeltétel teljesül akkor a ciklusinvariáns is)
			\item $ P \land \lnot \pi \implies R $ \\ (ha a ciklusinvariáns teljesül de a ciklusfeltétel nem akkor az utófeltétel teljesül)
			\item $ P \implies \pi \lor \lnot \pi $ \\ (ha a ciklusinvariáns teljesül akkor a ciklusfeltétel kiértékelhető)
			\item $ P \land \pi \implies t > 0 $ \\ (ha a ciklusinvariáns teljesül akkor a termináló függvény értéke pozitív)
			\item $ P \land \pi \implies \lf(S_0,P) $ \\ (ha a ciklusinvariáns és a ciklusfeltétel teljesül, akkor a program helyesen terminál úgy az invariáns igaz marad)
			\item $ P \land \pi \land t = t_0 \implies \lf(S_0,t<t_0) $ \\ (ha a ciklusinvariáns, a ciklusfeltétel és $t = t_0$ teljesül, akkor a program helyesen terminál úgy hogy csökken a termináló függvény értéke)
		\end{enumerate}
		akkor $ Q \implies \lf(DO,R) $
		\begin{remark}
			A tétel 5. és 6. pontjai összevonhatóak a következő módon:
			\[
			P \land \pi \land t = t_0 \implies \lf(S_0, t<t_0) \land lf (S_0,P) = \lf(S_0,P \land t<t_0) \]
		\end{remark}
	\end{theorem*}
		\newpage
\begin{question}(10. feladatsor)
	Lássa be, hogy az $S$ program megoldja a következő feladatot:
	\begin{align*}
		A &= \left( x: \mathbb{N}^+, l: \mathbb{L} \right) \\
		B &= \left( x': \mathbb{N}^+ \right) \\
		Q &= (x = x' \, \land \, x > 1) \\
		R &= (Q \, \land \, l = \left( \forall j \in \left[2..x-1\right]: j \centernot{|} x\right))
	\end{align*}
	Az program állapottere $ \left( x: \mathbb{N}^+, k: \mathbb{N}^+, l: \mathbb{L} \right) $.

	\begin{wrapfigure}[4]{l}{3.5cm}
			\vspace{-2\baselineskip}
		\begin{stuki*}[3.5cm]{S}
			\stm{k,l \coloneq 2,igaz}
			\begin{WHILE}{2}{\stm{k \ne x}}
				\stm{l \coloneq l \, \land \, k \centernot{|} x}
				\stm{k \coloneq k + 1}
			\end{WHILE}
		\end{stuki*}
	\end{wrapfigure}\leavevmode
	\noindent
	\\[18pt]
	$Q' = \left( Q \land k = 2 \land l = igaz \right) \text{ a szekvencia közbülső állítása}$
	\\[22pt]
	$Q'' = \left( Q \land l = \left( \forall j \in \left[ 2..k\right]: j \centernot{|} x  \right) \land k+1 \in \left[ 2..x \right] \land x-k=t_0 \right) $ a ciklusmag mint szekvencia közbülső állítása
	\\[-10pt]
	\noindent
	\begin{align*}
		\text{Legyen továbbá: } t &= x - k \text{ a termináló függvény és} \\
		P &= \left( Q \land l = \left( \forall j \in \left[ 2..k-1\right]: j \centernot{|} x \right) \land k \in \left[ 2..x \right]  \right) \text{a ciklusinvariáns.}
	\end{align*}
	\end{question}
	\begin{solution}
		A szekvencia levezetési szabálya alapján a program helyes, ha:
		\begin{enumerate}
			\item {$ Q \implies \lf(S_1, Q') $
				\begin{align*}
					Q &\implies \lf(k,l \coloneq 2,igaz; \left( Q \land k = 2 \land l = igaz \right)) \\
					Q &\implies Q^{k \leftarrow 2, l \leftarrow igaz} \\
					Q &\implies Q \land 2 = 2 \land igaz = igaz \text{ ami nyilván teljesül}
				\end{align*}
			}
			\item $ Q' \implies \lf(S_2,R) $ $S_2$ egy ciklus ezért a ciklus levezetési szabályát kell használnunk:
			\begin{enumerate}
				\item $ Q' \implies P $
				\begin{align*}
					\left( Q \land \underline{k = 2 \land l = igaz} \right) &\implies \left( Q \land l = \left( \forall j \in \left[ 2..k-1\right]: j \centernot{|} x \right) \land k \in \left[ 2..x \right]  \right) \\
					Q' &\implies ( Q \land igaz = ( \underbrace{\forall j \in \left[ 2..1\right]: j \centernot{|} x }_{\forall x \in \emptyset: \dots \iff igaz}) \land \underbrace{ 2 \in \left[ 2..x \right]}_{\substack{Q \text{ miatt } x>1 \\ \text{ezért } \forall x \in \mathbb{N}: \\ 2 \in [2..x]}})
				\end{align*}
				\item $ (P \land \lnot \pi) \implies R $
				\begin{align*}
					( Q \land l &= ( \forall j \in \left[ 2..k-1\right]: j \centernot{|} x ) \land k \in \left[ 2..x \right] \land \ \underline{k = x} ) \implies R \\
					(Q \land l &= ( \forall j \in \left[ 2..x-1\right]: j \centernot{|} x ) \land \underbrace{x \in \left[ 2..x \right]}_{\iff x>1}  )\implies R \\
					(Q \land l &= ( \forall j \in \left[ 2..x-1\right]: j \centernot{|} x )) \implies (Q \, \land \, l = ( \forall j \in \left[2..x-1\right]: j \centernot{|} x)) 
				\end{align*}
				\item $ P \implies \pi \lor \lnot \pi $
				Mivel $k,x \in \mathbb{N}^{+}$ ezért minden esetben kiértékelhetők, ezért az állítás teljesül.
				\item $ P \land \pi \implies t > 0 $
				\begin{align*}
					( Q \land l = \left( \forall j \in \left[ 2..k-1\right]: j \centernot{|} x \right) \land k \in \underline{\left[ 2..x \right]} \land \underline{x \ne k} ) &\implies x-k > 0 \\
					( Q \land l = \left( \forall j \in \left[ 2..k-1\right]: j \centernot{|} x \right) \land k \in \underbrace{\left[ 2..x-1 \right]}_{x>k} ) &\implies x-k > 0
				\end{align*}
				Mivel $x > k$ ezért $x - k > 0$
				\item $ P \land \pi \land t = t_0 \implies \lf{(S_0, P \land t < t_0)} $, ahol $S_0$ az $S_2$ ciklusmagja
				\begin{align*}
					(P \land x \ne k \land x - k = t_0) \implies \lf{(S_0, P \land x-k < t_0)}
				\end{align*}
				$S_0$ egy szekvencia ezért a következőket kell belátni:
				\begin{enumerate}
					\item $ (P \land x \ne k \land x - k = t_0) \implies \lf((l \coloneq l \land k \centernot{|} x), Q'') $
					\begin{align*}
						&(P \land x \ne k \land x - k = t_0) \implies Q''^{l \leftarrow l \land k \centernot{|} x} \\
						&\left( \underline{Q} \land l = \left( \forall j \in \left[ 2..k-1\right]: j \centernot{|} x \right) \land k \in \left[ 2..x \right]   \land x \ne k \land \underline{x - k = t_0} \right) \\
						&\implies \left( \underline{Q} \land (l \land k \centernot{|} x) = \left( \forall j \in \left[ 2..k\right]: j \centernot{|} x  \right) \land k+1 \in \left[ 2..x \right] \land \underline{x-k=t_0} \right)
					\end{align*}
					Átalakítva az $\forall j \in \left[ 2..k\right]$ tagot:					
					\begin{align*}
						(l \land k \centernot{|} x) = \left( (\forall j \in \left[ 2..k-1\right]: j \centernot{|} x) \land k \centernot{|} x  \right) \land k+1 \in \left[ 2..x \right]
					\end{align*}
					Az invariánsból tudjuk, hogy $ l = \left( \forall j \in \left[ 2..k-1\right]: j \centernot{|} x \right) $, ezért:
					\begin{align*}
						(l \land k \centernot{|} x) = \left( l \land k \centernot{|} x  \right) \land k+1 \in \left[ 2..x \right]
					\end{align*}
					Így már csak azt kell belátni, hogy:
					\begin{align*}
						k \in [2..x] \land k \ne x &\implies k + 1 \in [2..x] \\
						k \in [2..x-1] & \implies k + 1 \in [3..x]
					\end{align*}
					Ami igaz mert $[3..x] \subset [2..x]$.
					\item $Q'' \implies \lf{(k \coloneq k+1, P \land x-k < t_0)} $
					\begin{align*}
						Q'' &\implies (P \land x - k < t_0)^{k \leftarrow k+1} \\
						Q'' &\implies  ( \underbrace{( Q \land l = \left( \forall j \in \left[ 2..k\right]: j \centernot{|} x \right) \land k+1 \in \left[ 2..x \right]  )}_{Q'' \text{ része}} \land x - k - 1 < t_0 )
					\end{align*}
					Tudjuk továbbá, $Q''$ miatt, hogy $x-k = t_0$ így:
					\begin{align*}
						x-k-1 < t_0 \iff t_0-1 < t_0
					\end{align*}
					ami nyilván igaz.
				\end{enumerate}
			\end{enumerate}
		\end{enumerate}
		Így a specifikáció tétele alapján $S$ megoldja a feladatot.
	\end{solution}
\end{document}