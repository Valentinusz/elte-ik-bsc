\begin{definition*}[ábécé]
	Szimbólumok egy véges nemüres halmaza. Például. $ V = \left\lbrace a,b \right\rbrace $.
\end{definition*}

\begin{definition*}[szimbólum]
	Egy tetszőleges $ V $ ábécé elemeit szimbólumoknak vagy betűknek nevezzük.
\end{definition*}

\begin{definition*}[szó]
	Egy $ u \in V^* $ (V ábécé elemiből álló véges) sorozatot $ V $ feletti szónak (vagy sztringnek) nevezünk.
\end{definition*}

\section{Szavak}
\subsection{Alapfogalmak}

\begin{definition*}
	Legyen $ u \in V^* $ egy szó, ekkor a benne lévő betűk számát $u$ hosszának nevezzük  és $ l(u) $-val vagy $\left| u \right| $-el jelöljük.
\end{definition*}

\begin{notation*}
	Egy $ \delta \in V $ betű az $ u \in V^* $ szóban lévő előfordulásinak számát $ l(u)_{\delta} $-val vagy $ \left| u \right|_{\delta} $-val jelöljük.
\end{notation*}

\begin{definition*}[üres szó]
	Legyen $V$ egy ábécé, ekkor üres szónak nevezzük azt az $ \varepsilon $ szót melyre $ \left| \varepsilon \right| = 0 $. 
\end{definition*}

\begin{remark}
	Világos, hogy $ \varepsilon \in V^* $  bármely $ V $ abécé esetén.
\end{remark}

\begin{definition*}[$ V^+ $]
	Tetszőleges $ V $ ábécé esetén $ V^+ $ jelöli az $ V $ feletti nemüres szavak halmazát, azaz a $ V^+ = V^* \setminus \left\lbrace \varepsilon \right\rbrace $ halmazt. 
\end{definition*}

\subsection{Műveletek}
\subsubsection{Konkatenáció}
\begin{definition*}
	Legyen $V$ egy ábécé és legyenek $ u = s_1 \dots s_n $ és $ v = t_1 \dots t_k $ $V$ feletti szavak. Ekkor az $uv \coloneq s_1 \dots s_n t_1 \dots t_k $ szót $u$ és $v$ konkatenáltjának nevezzük.
\end{definition*}
\paragraph{Tulajdonságok}
\begin{enumerate}
	\item{$ \left| uv \right| = \left| u \right| + \left| v \right| $}
	\item általában nem kommutatív (kivétel egyetlen betűből álló ábécék)
	\item asszociatív: $ u,v,w \in V^* \implies u(vw) = (uv)w $
	\item $ \forall u,v \in V^*: uv \in V^* $ ($ V^* $ a konkatenáció műveletére zárt)
	\item létezik egységelem: $ \forall u \in V^*: u\varepsilon = u $
\end{enumerate}
Így $V^*$ félcsoport.

\newpage
\subsubsection{Hatványozás}
\begin{definition*}
	Legyen $ i \in \mathbb{N}^+ $ és $ u \in V^* $. Ekkor $u$ $i$-edik hatványának nevezzük az $u$ szó $i$ darab példányának konkatenáltját.
	\[
	u^0 = \varepsilon, \; u^i = uu^{i-1} \, (i \in \mathbb{N}^+)
	\]
\end{definition*}

\begin{remark}
	Nyilván $\varepsilon^0 = \varepsilon$.
\end{remark}
\paragraph{Tulajdonságok}
\begin{enumerate}
	\item{$u^{n+k} = u^n u^k \; \left( k,n \in \mathbb{N} \right)  $}
	\item{$ (ab)^n \ne a^nb^n $}
\end{enumerate}

\subsubsection{Tükörkép}
\begin{definition*}
	Legyen $ u = a_1 \dots a_n $, ekkor a szó tükörképének (megfordítottjának) nevezzük a
	\[
	u^R = a_n \dots a_1 \; (1 \le i \le n: u_i = u^R_{n+1-i})
	\]
	szót. Alternatív jelölés: $u^{-1}$, rev($ u $).
\end{definition*}
\begin{remark}
	Ha $u = u^R$ a szót palindrómának (vagy palindrom tulajdonságúnak) nevezzük.
\end{remark}

\paragraph{Tulajdonságok}\mbox{}\\[-20pt]

\begin{multicols}{2}
	\begin{enumerate}
		\item{$ \varepsilon^R = \varepsilon $}
		\item{$ \left(u^R \right)^R = u $}
		\item{$ \left( uv \right)^R = v^R u^R $}
		\item{$ \left( u^i \right)^R = \left( u^R \right)^i \; (i \in \mathbb{N}) $}
	\end{enumerate}
\end{multicols}
\subsection{Résszavak}
\subsection{Résszó}
Legyen $V$ egy ábécé és legyenek $u$ és $v$ szavak $V$ felett.
\begin{definition*}[résszó]
	Az $u$ szó résszava a $v$ szónak, ha $ \exists x,y \in V^*: v = xuy $.
\end{definition*}
\begin{definition*}[valódi résszó]
	Az $u$ szó valód résszava a $v$ szónak, ha résszó és $ u \ne v $ és $ u \ne \varepsilon $.
\end{definition*}
\begin{definition*}[prefixum]
	Ha, $ v = xuy$, úgy hogy $x = \varepsilon$, akkor $u$-t $v$ prefixumának nevezzük.
\end{definition*}
\begin{definition*}[szuffixum]
	Ha, $ v = xuy$, úgy hogy $y = \varepsilon$, akkor $u$-t $v$ szuffixumának nevezzük.
\end{definition*}
\noindent
Az $u$ szót $v$ valódi prefixumainak/szuffixumainak nevezzük, ha $u \ne \varepsilon \land u \ne v $.

\section{Nyelv}
\begin{definition*}[nyelv]
	Legyen $V$ egy ábécé, ekkor nyelvnek nevezzük az $ L \subseteq V^* $ halmazt. Ekkor $L$-t $V$
\end{definition*}

\begin{notation*}
	$ \emptyset $-el jelöljük az üres nyelvet. $\emptyset \ne \left\lbrace \varepsilon \right\rbrace $ 
\end{notation*}

\subsection{Műveletek}
\noindent
Mivel a nyelvek halmazok értelmezzük az unió, metszet, különbség és komplementer műveleteket. 
\subsubsection{Konkatenáció}

\begin{definition*}
	Legyen $V^*$ egy ábécé és $L_1, L_2 \subseteq V^*$, ekkor $L_1$ és $L_2$ konkatenációjának nevezzük az 
	\[
	L_1L_2 = \left\lbrace u_1,u_2 | u_1 \in L_1, u_2 \in L_2 \right\rbrace 
	\]
	a nyelvet.
\end{definition*}
\begin{example}
	\[
	\left\lbrace a,b \right\rbrace \left\lbrace ab,b \right\rbrace = \left\lbrace aab, ab, bab, bb \right\rbrace   
	\]
\end{example}
\paragraph{Tulajdonságok}
\begin{enumerate}
	\item Minden $ L $ nyelv esetén: $ \left\lbrace \varepsilon \right\rbrace L = L \left\lbrace \varepsilon \right\rbrace $
	\item Asszociatív
	\item Egységelem: $ \left\lbrace \varepsilon \right\rbrace $.
\end{enumerate}

\subsubsection{Hatványozás}
\begin{definition*}
	Legyen $V^*$ egy ábécé és $L \subseteq V^* $, ekkor $L$ $i$-edik hatványának nevezzük a
	\[
	L^0 = \left\lbrace \varepsilon \right\rbrace, \qquad L^i = LL^{i-1} \quad (i \ge 1)
	\]
	a nyelvet.
\end{definition*}
\begin{remark}
	$ \emptyset^0 = \left\lbrace \varepsilon \right\rbrace $. 
\end{remark}
\subsubsection{Iteratív lezárt}
\begin{definition*}
	Egy $L$ nyelv iteratív lezártjának nevezzük az
	\[
	L^* = \bigcup_{i \ge 0}{L^i}=L^0 \cup L^1 \cup \dots
	\]
	nyelvet.
\end{definition*}

\subsubsection{Pozitív lezárt}
\begin{definition*}
	Egy $L$ nyelv pozitív lezártjának nevezzük az
	\[
	L^+ = \bigcup_{i \ge 1}{L^i}=L^0 \cup L^1 \cup \dots = L^* \setminus \left\lbrace \varepsilon \right\rbrace 
	\]
	nyelvet.
\end{definition*}
\newpage
\subsection{Feladatok}
\subsubsection{}
Legyenek
\begin{align*}
	L_1 &= \left\lbrace a, bb, aba \right\rbrace \\
	L_2 &= \left\lbrace ab^n \; | \; n \ge 0 \right\rbrace = \left\lbrace a, ab, abb, \dots \right\rbrace  \\
	L_3 &= \left\lbrace u \in \left\lbrace a,b \right\rbrace ^* \; | \; l_a(u) = l_b(u) \right\rbrace
	= \left\lbrace \varepsilon, ab, ba, \dots \right\rbrace  \\
	L_4 &= \left\lbrace u \in \left\lbrace a,b \right\rbrace ^* \; | \; l_b(u) \bmod 2  = 0 \right\rbrace
	= \left\lbrace \varepsilon, a, bb, abb, aabb, \dots \right\rbrace \\
	L_5 &= \left\lbrace \varepsilon, ba \right\rbrace
\end{align*}
nyelvek. Határozzuk meg:
\begin{align*}
	L_1 \cap L_2 &= \left\lbrace a \right\rbrace \\
	L_1 \cap L_3 &= \emptyset \\
	L_1 \cap L_4 &= \left\lbrace a, bb \right\rbrace \\
	L_2 \setminus L_1 &= \left\lbrace ab^n \; | \; n \ge 1 \right\rbrace \\
	L_1L_5 &= \left\lbrace a, aba, bb, bbba, ababa \right\rbrace \\
	L_1^2 &= \left\lbrace aa, abb, aaba, bba, bbbb, bbaba, abaa, ababb, abaaba \right\rbrace 
\end{align*}
\subsubsection{}
Legyenek
\begin{align*}
	L_1 &= \left\lbrace a^nb^m \; | \; m \ge n \land n \ge 0 \right\rbrace = \left\lbrace \varepsilon, b, ab, abb, \dots \right\rbrace  \\
	L_2 &= \left\lbrace ab \right\rbrace^* = \left\lbrace \varepsilon, ab, abab, \dots \right\rbrace 
\end{align*}
nyelvek. Határozzuk meg:
\begin{align*}
	L_1 \cap L_2 &= \left\lbrace \varepsilon, ab \right\rbrace \\
	L_1 \setminus L_2^* &= \left\lbrace a^nb^m \; | \; m \ge n \ge 2 \right\rbrace \cup \left\lbrace b \right\rbrace^+ \cup \left\lbrace ab^n \; | \; n \ge 1 \right\rbrace \\
	L_1^* &= \left\lbrace \varepsilon, b, ab, abb, bb, bab, abab, \dots \right\rbrace \\
	L_2 \ L_1^* &= \emptyset \qquad (ab \in L_1^* \text{ miatt}) \\
	L_2^* &= L_2
\end{align*}