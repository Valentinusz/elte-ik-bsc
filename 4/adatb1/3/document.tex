\documentclass[a4paper,12pt]{article}

\usepackage[margin=0.75in]{geometry}
\usepackage[utf8]{inputenc}
\usepackage{exsheets}
\usepackage{lmodern}
\usepackage[T1]{fontenc}

\DeclareInstance{exsheets-heading}{block-no-nr}{default}{
	attach = {
		main[l,vc]title[l,vc](0pt,0pt) ;
		main[r,vc]points[l,vc](\marginparsep,0pt)
	}
}
\RenewQuSolPair
{question}[headings=runin]
{solution}[headings=block-no-nr]

\SetupExSheets{
	counter-format=qu.,
	solution/print=true ,
	question/name=Feladat,
	solution/name=Megoldás.
}

\usepackage[hungarian]{babel}
\usepackage{amsmath}
\usepackage{mathtools}
\usepackage{amsthm}
\usepackage{amsfonts}
\usepackage{amssymb}
\usepackage{graphicx}
\usepackage{wrapfig}
\graphicspath{{./images/}}
\usepackage{float}
\usepackage{multicol}

\usepackage{titling}
\setlength{\droptitle}{-2cm}

\title{\huge{Adatbázisok I.} \\[-4pt] \large 3. gyakorlat \vspace{-15pt}}
\author{Boda Bálint}
\date{\vspace{-12pt}2023. tavaszi félév}

\DeclareMathOperator{\lf}{lf}

\theoremstyle{definition}
\newtheorem{definition}{Definíció}
\newtheorem*{definition*}{Definíció}
\newtheorem*{remark}{Megjegyzés}
\newtheorem{theorem}{Tétel}
\newtheorem*{theorem*}{Tétel}
\newtheorem*{example}{Példa}
\newtheorem{notation}{Jelölés}
\newtheorem*{notation*}{Jelölés}

\SetupExSheets{solution/print=true}
\SetupExSheets{question/name=}
\SetupExSheets{headings=runin}

\begin{document}
\maketitle
\section*{Relációs algebra}

Műveletek:
\begin{itemize}
	\item {
		$ \pi_{A,B}{\left( R \right) } $ projekció, egy olyan relációt ad vissza, mely $R$ adott attribútumainak levetítéséből áll
	}
	\item {
		$ \sigma_{P}{\left( R \right)} $ szelekció, egy olyan relációt ad vissza, mely $R$ azon sorait tartalmazza, melyek megfelelnek a $P$ predikátumnak
	}
	\item {
		$ \rho_{(S_{(C,B)})}{\left( R \right)} $ átnevezés, egy olyan $S$ relációt ad vissza, mely sorai megegyeznek $R$ soraival, de attribútumai pedig rendre $ C,B $
	}
	\item {
		halmazműveletek, Descartes-szorzat
	}
\end{itemize}
\noindent
Tekintsük a $ szeret $ táblát:
\begin{table}[H]
	\centering
	\begin{tabular}{|c|c|}
		\hline
		Név & Gyümölcs \\
		\hline
		Micimackó & Alma \\
		\hline
		Micimackó & Körte \\
		\hline
		Tigris & Alma \\
		\hline
		Bagoly & Eper \\
		\hline
	\end{tabular}
\end{table}
	
\noindent
Adjuk meg azokat a gyümölcsöket melyeket Micimackó szeret!
\[
\pi_{gyümölcs}{\left(  \sigma_{név = Micimackó}{\left( szeret  \right) }\right) }
\]
Adjuk meg azokat a gyümölcsöket melyeket Micimackó nem szeret!
\[
\pi_{gyümölcs}{\left( szeret \right)} \setminus \pi_{gyümölcs}{\left(  \sigma_{név = Micimackó}{\left( szeret  \right) }\right) }
\]
\end{document}