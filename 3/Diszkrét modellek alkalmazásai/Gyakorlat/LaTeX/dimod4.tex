\documentclass[a4paper,12pt]{article}
\usepackage[utf8]{inputenc}
\usepackage{exsheets}

\DeclareInstance{exsheets-heading}{block-no-nr}{default}{
	attach = {
		main[l,vc]title[l,vc](0pt,0pt) ;
		main[r,vc]points[l,vc](\marginparsep,0pt)
	}
}

\RenewQuSolPair
{question}[headings=runin]
{solution}[headings=block-no-nr]

\SetupExSheets{
	counter-format=qu.,
	solution/print=true ,
	question/name=Feladat,
	solution/name=Megoldás.
}

\usepackage{tasks}
\usepackage[hungarian]{babel}
\usepackage{amsmath}
\usepackage{mathtools}
\usepackage{amsthm}
\usepackage[shortlabels]{enumitem}
\usepackage{amsfonts}
\usepackage{amssymb}
\usepackage{graphicx}
\usepackage{wrapfig}
\graphicspath{{./images/}}
\usepackage{float}
\usepackage{multicol}
\usepackage{tikz}
\usepackage{minted}
\usetikzlibrary{positioning,shapes,fit,arrows}
\tikzset{every picture/.style={line width=0.75pt}} %set default line width to 0.75pt    

\title{\huge{Diszkrét modellek alkalmazásai} \\ \large 4. gyakorlat}
\author{Boda Bálint}
\date{2022. őszi félév}

\DeclareMathOperator{\interior}{int}
\DeclareMathOperator{\lf}{lf}
\DeclareMathOperator{\lnko}{lnko}

\theoremstyle{definition}
\newtheorem{definition}{Definíció}
\newtheorem*{definition*}{Definíció}
\newtheorem*{remark}{Megjegyzés}
\newtheorem{theorem}{Tétel}
\newtheorem*{theorem*}{Tétel} 
\newtheorem*{example}{Példa}

\SetupExSheets{solution/print=true}
\SetupExSheets{question/name=}
\SetupExSheets{headings=runin}

\begin{document}
	\maketitle
	\section{Euklideszi algoritmus}
	Az Euklideszi algoritmus egy optimális mód az $a,b \in \mathbb{Z}$ számok legnagyobb közös osztójának meghatározására.
	Ha $a < b$, akkor felcseréljük a két számot majd addig ismételjük a következő lépést amíg az $r_i$ osztási maradék $0$ nem lesz.
	\begin{align*}
		a&=q_{0} \cdot b+r_{0}\\
		b&=q_{1} \cdot r_{0}+r_{1}\\
		r_{0}&=q_{2} \cdot r_{1}+r_{2}\\
		r_{1}&=q_{3} \cdot r_{2}+r_{3}\\
		&\dotsb
	\end{align*}
	Ekkor $\lnko{(a,b)} = r_{i-1}$.
	\begin{example} $ lnko(360,225) = 45 $
		\begin{align*}
		360&=1 \cdot 225+135\\
		225&=1 \cdot 135+90\\
		135&=1 \cdot 90+\mathbf{45}\\
		90&=2 \cdot 45+0\\
		\end{align*}
	\newpage
		\begin{question}
Számítsuk ki a következő számok legnagyobb közös osztóját!
			\begin{tasks}(3)
				\task 30 és 70
				\task 126 és 150
				\task 105 és 231
				\task 132 és 275
				\task 33 és 21
			\end{tasks}
		\end{question}
		\begin{solution}
			\begin{tasks}(3)
				\task $\lnko{(30,70)} = 10 $ \begin{align*}
					70&=2 \cdot 30+\mathbf{10}\\
					30&=3 \cdot 10+0
				\end{align*}
				\task $\lnko{(126,150)} = 6 $ \begin{align*}
					150&=1 \cdot 126+24\\
					126&=5 \cdot 24+\mathbf{6}\\
					24&=4 \cdot 6 + 0
				\end{align*}
				\task $\lnko{(105,231)} = 21 $ \begin{align*}
					231&=2 \cdot 105+\mathbf{21}\\
					105&=5 \cdot 21+0
				\end{align*}
					\task $\lnko{(132,275)} = 11 $ \begin{align*}
					275&=2 \cdot 132+\mathbf{11}\\
					132&=12 \cdot 11+0
				\end{align*}
				\task $\lnko{(33,21)} = 3 $ \begin{align*}
					33&=1 \cdot 21+12\\
					21&=1 \cdot 12 + 9\\
					12&=1 \cdot 9 + \mathbf{3}\\
					9&=3 \cdot 3 + 0
				\end{align*}
			\end{tasks}
		\end{solution}
	\end{example}
	\subsection{Python nyelven}
	\begin{minted}[tabsize=4]{python3}
def lnko(a,b):
	if a == b:
		return a
	if a < b:
		a, b = b, a
	while (b > 0):
		a, b = b, a % b
	return a
	\end{minted}
\newpage
\section{Kongruencia}
\begin{definition*}[Kongruencia]
	Kongruenciának nevezzük a $$ \{(a,b) \in \mathbb{Z} \times \mathbb{Z} \mid a \bmod m =\; b \bmod m \quad (m \in \mathbb{Z})\} $$ $ a \equiv b \pmod m$-el (ejtsd: $a$ kongruens $b$ modulo $m$) jelölt relációt.
\end{definition*}
\begin{theorem*}
	A kongruencia ekvivalenciareláció, ekvivalenciaosztályait pedig \textbf{maradékosztályoknak} nevezzük.
\end{theorem*}

\begin{example} $ $ \\[-20pt]
	\begin{itemize}
		\item $ 5 \equiv 11 \pmod 6 $
		\item $ 2 \not \equiv 6 \pmod 3 $
		\item $ -8 \equiv 10 \pmod 6 = 0 $, mert $ -2 \cdot 6+4 = -8$ és $ 1 \cdot 6 + 4 = 10$
	\end{itemize}
\end{example}

\begin{question}
	Igazak-e a következő kongruenciák?
	\begin{tasks}(3)
		\task $ 7 \equiv 3 \pmod 3 $
		\task $ 7 \equiv 3 \pmod 2 $
		\task $ 7 \equiv 3 \pmod 1 $
		\task $ 8 \equiv 10 \pmod 5 $
		\task $ 2 \equiv -1 \pmod 3 $
		\task $ 6 \equiv 6 \pmod{100} $
		\task $ 11 \equiv 8 \pmod 3 $
		\task $ 8 \equiv 5 \pmod 3 $
		\task $ 11 \equiv 5 \pmod 3 $
		\task $ 6 \equiv 2 \pmod 4 $
		\task $ 3 \equiv -5 \pmod 4 $
		\task $ 18 \equiv -10 \pmod 4 $
		\task $ 160 \equiv 80 \pmod{16} $
		\task $ 16 \equiv 8 \pmod 8 $
	\end{tasks}
\end{question}

\begin{solution}
	\begin{tasks}(3)
		\task hamis ($ 1 \ne 0 $)
		\task igaz ($ 1 = 1 $)
		\task hamis ($ 0 = 0 $)
		\task hamis ($ 3 \ne 0 $)
		\task igaz ($ 2 = 2 $)
		\task igaz ($ 6 = 6 $)
		\task igaz ($ 2 = 2 $)
		\task igaz ($ 2 = 2 $)
		\task igaz ($ 2 = 2 $)
		\task igaz ($ 2 = 2 $)
		\task igaz ($ 3 = 3 $)
		\task igaz ($ 2 = 2 $)
		\task igaz ($ 0 = 0 $)
		\task igaz ($ 0 = 0 $)
	\end{tasks}
\end{solution}

\subsection{Lineáris kongruenciák}
Lineáris kongruenciának nevezzük az $ ax \equiv b \pmod m $ alakú kongruenciákat.
\begin{theorem*}
	Egy $ ax \equiv b \pmod m$ kongruencia, akkor oldható meg, ha $$ \lnko{(a,m)} \mid b $$ 
\end{theorem*}

\begin{example}
	$$ x \equiv 5 \pmod 7 $$
	Mivel $ \lnko(1,7) = 1 $ osztója $5$-nek ezért a kongruenciaegyenlet megoldható és megoldásai az $5 + 7t \; (t \in \mathbb{Z})$ alakú egész számok.
	\begin{remark}
		A megoldások halmaza, az $ x \equiv 5 \pmod 7 $ reláció, azon ekvivalenciaosztálya, melynek eleminek $7$-el vett osztási maradéka $5$.
	\end{remark}
\end{example}
\subsubsection{Kongruencia azonosságai}
\begin{itemize}
	\item A kongruenciához szabadon hozzáadhatunk és kivonhatunk. $$ x + 4 \equiv 5 \pmod 7 \iff x \equiv 1 \pmod 7 $$
	\item A kongruencia egyik oldalához hozzáadhatjuk vagy kivonhatjuk $m$-et. $$ x \equiv 12 \pmod 7 \iff x \equiv 5 \pmod 7 $$
	\item A kongruenciát megszorozhatunk egy tetszőleges $ k \in \mathbb{Z} $ számmal. $$ x \equiv 4 \pmod 7 \iff 2x \equiv 8 \pmod 7 $$
	\item A kongruenciát leoszthatjuk egy tetszőleges $ k \in \mathbb{Z} $ számmal, de ekkor a modulust is osztani kell. $$ 4x \equiv 8 \pmod{14} \iff x \equiv 2 \pmod 7 $$ $$ ax \equiv b \pmod m \iff x \equiv \frac{b}{a} \left(\bmod{\frac{m}{\lnko{(a,m)}}}\right)  $$
\end{itemize}
\newpage
\begin{question}
	Oldja meg a következő kongruenciaegyenleteket!
	\begin{tasks}(3)
		\task $ 2x \equiv 3 \pmod{4} $
		\task $ x \equiv 2 \pmod{3} $
		\task $ x \equiv 7 \pmod{2} $
		\task $ 12x \equiv 8 \pmod{20} $
		\task $ 22x \equiv 8 \pmod{10} $
		\task $ 15x \equiv -1 \pmod{7} $
	\end{tasks}
\end{question}
\begin{solution}
	\begin{tasks}
		\task $ 2x \equiv 3 \pmod{4}$, mivel $\lnko(2,4) = 2 \nmid 3 $, ezért nincs megoldás.
		\task $ x \equiv 2 \pmod{3}$, $\lnko{(1,3)} = 1 \mid 2 $, ezért a kongruencia megoldható és megoldásai a $ 2 + 3t \; (t \in \mathbb{Z}) $ alakú számok.
		\task $ x \equiv 7 \pmod{2}$,	$\lnko{(1,2)} = 1 \mid 7$ \\ $ x = 7 + 2t \, (t \in \mathbb{Z}) $
		\task{$ 12x \equiv 8 \pmod{20} \qquad\lnko{(12,20)} = 4 \mid 8 $
			\begin{alignat*}{2}
				12x \equiv 8 \pmod{20} &\iff 12x \equiv 48 \pmod{20} \\
				&\iff x \equiv 4 \pmod{\frac{20}{lnko(12,20)}=\frac{20}{4}=5} 
			\end{alignat*} $ x = 4 + 5t \; (t \in \mathbb{Z}) $}
		\task{ $ 22x \equiv 8 \pmod{10} \qquad\lnko{(12,10)} = 2 \mid 8 $
			\begin{alignat*}{2}
				22x \equiv 8 \pmod{20} &\iff 22x \equiv 88 \pmod{10} \\
				&\iff x \equiv 4 \pmod{\frac{10}{2}=5} 
			\end{alignat*} $ x = 4 + 5t \; (t \in \mathbb{Z}) $}
		\task{ $ 15x \equiv -1 \pmod{7} \qquad\lnko{(15,7)} = 1 \mid -1 $
			\begin{alignat*}{2}
				15x \equiv -1 \pmod{7} &\iff 15x \equiv 6 \pmod{7} \\
				&\iff 15x \equiv 90 \pmod{7} \\
				&\iff x \equiv 6 \pmod{7}  
			\end{alignat*} $ x = 6 + 7t \; (t \in \mathbb{Z}) $}
	\end{tasks}
\end{solution}
\end{document}